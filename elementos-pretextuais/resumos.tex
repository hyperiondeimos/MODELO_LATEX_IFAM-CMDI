% ----------------------------------------------------------
% RESUMOS
% ----------------------------------------------------------

% resumo em português
\setlength{\absparsep}{18pt} % ajusta o espaçamento dos parágrafos do resumo
\begin{resumo}
 Em virtude do aumento exponencial das frotas de carros e a ampla necessidade em ter um maior controle de acesso de automóveis a ambientes fechados, é importante que sistemas de reconhecimento automático de placas veiculares possam ser utilizados de forma simples e que controlem o acesso de veículos a ambientes fechados, apenas necessitando da supervisão humana. Atualmente, torna-se viável implementar esses tipos de sistemas utilizando uma tecnologia de baixo custo. Este trabalho apresenta um método de reconhecimento automático de placas de veículos com base em reconhecimento de padrões. Utilizando um microcomputador Raspberry Pi 3 como unidade controladora do sistema, uma \textit{webcam}, um sensor ultrassônico e um motor DC de um drive de CD-ROM, para simular o portão eletrônico. Após a aquisição da imagem, o sistema segue as seguintes etapas: pré-processamento da imagem, extração da área da placa, reconhecimento dos caracteres utilizando o classificador kNN e a tomada de decisão para o acionamento do portão eletrônico. O resultado obteve um índice de reconhecimento de 92\% das placas as quais foram submetidos nos testes e um bom desempenho do sistema completo em funcionamento sendo executado em cerca de 10 segundos para a realização do processo.

 \textbf{Palavras-chaves}: Reconhecimento de Caracteres. Reconhecimento de Padrões. Visão Computacional. Classificação. Controle.
\end{resumo}

% resumo em inglês
\begin{resumo}[Abstract]
 \begin{otherlanguage*}{english}
Due to the exponential increase in car fleets and the widespread across many cities, security is needed for a greater access control of cars parking garages. It's important that automatic vehicle-plate recognition systems can be used to control vehicle access to these environments, only needing human supervision. Currently, it's feasible to implement these types of systems using a low-to-zero cost technology. This work presents a method of automatic recognition of vehicle-plates based on pattern recognition. Using a Raspberry Pi 3 microcomputer as a system controller unit, a webcam, an ultrasonic sensor and a DC motor (from a CD-ROM drive) to simulate the electronic gate, is possible to create a system capable of using the vision sensor to direct feed the source code with a vehicle-plate to check if that car with same plate can be allowed to enter in parking garage or not. After the image acquisition, the system follows the following steps: Image pre-processing, plate extraction area, character recognition using the kNN classifier and electronic gate operation through a decision making. The result shows a recognition success rate of 92\% of the plates, to which the tests were submitted and a reasonable performance of the complete system in operation, being executed in approximally 10 seconds for the accomplishment of the complete process.

   \vspace{\onelineskip}
 
   \noindent 
   \textbf{Key-words}: Characters Recognition. Pattern Recognition. Computer Vision. Classification. Control.
 \end{otherlanguage*}
\end{resumo}

% Só descomente se for necessário

% resumo em francês 
%\begin{resumo}[Résumé]
% \begin{otherlanguage*}{french}
%    Il s'agit d'un résumé en français.
% 
%   \textbf{Mots-clés}: latex. abntex. publication de textes.
% \end{otherlanguage*}
%\end{resumo}
%
%% resumo em espanhol
%\begin{resumo}[Resumen]
% \begin{otherlanguage*}{spanish}
%   Este es el resumen en español.
%  
%   \textbf{Palabras clave}: latex. abntex. publicación de textos.
% \end{otherlanguage*}
%\end{resumo}

