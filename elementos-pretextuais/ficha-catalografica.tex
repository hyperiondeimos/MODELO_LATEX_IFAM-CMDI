% ----------------------------------------------------------
% FICHA BIBLIOGRAFICA
% ----------------------------------------------------------

% Isto é um exemplo de Ficha Catalográfica, ou ``Dados internacionais de
% catalogação-na-publicação''. Você pode utilizar este modelo como referência. 
% Porém, provavelmente a biblioteca da sua universidade lhe fornecerá um PDF
% com a ficha catalográfica definitiva após a defesa do trabalho. Quando estiver
% com o documento, salve-o como PDF no diretório do seu projeto e substitua todo
% o conteúdo de implementação deste arquivo pelo comando abaixo:
%
% \begin{fichacatalografica}
%     \includepdf{fig_ficha_catalografica.pdf}
% \end{fichacatalografica}
\begin{fichacatalografica}
	\vspace*{\fill}					% Posição vertical
	\hrule							% Linha horizontal
	\begin{flushleft}
		R696m
	\end{flushleft}
	\begin{center}					% Minipage Centralizado
	\begin{minipage}[c]{12.5cm}		% Largura
	
	\imprimirautor
	
	\hspace{0.5cm} \imprimirtitulo  / \imprimirautor. --
	\imprimirlocal, \imprimirdata-
	
	\hspace{0.5cm} \pageref{LastPage} p. : il. (algumas color.) ; 30 cm.\\
	
	\hspace{0.5cm} \imprimirorientadorRotulo~\imprimirorientador\\
	
	\hspace{0.5cm}
	\parbox[t]{\textwidth}{\imprimirtipotrabalho~--~\imprimirinstituicao,
	\imprimirdata.}\\
	
	\hspace{0.5cm}
		1. Reconhecimento de Caracteres.
		2. Reconhecimento de Padrões.
		I. Anderson Gadelha Fontoura.
		II. Centro Universitário do Norte - UNINORTE.
		III. Curso de Engenharia Elétrica.
		IV. Melhoria de Segurança em Portões Eletrônicos Utilizando Reconhecimento de Padrões.\\ 			
	
	\hspace{8.75cm} CDU 621.319\\
	
	\end{minipage}
	\end{center}
	\hrule
\end{fichacatalografica}
