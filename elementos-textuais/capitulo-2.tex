\chapter{\textbf{\uppercase{Fundamentação Teórica}}}\label{cap_2}

Neste capitulo será contextualizado as informações teóricas para a compreensão deste projeto. O capitulo deve prover ao leitor a base teórica para compreender os conceitos que serão melhor explicados ao longo deste capítulo para poder familiarizar com os futuros tópicos vistos em outros capítulos.

\section{\uppercase{Visão Computacional}}

A visão computacional busca igualar ou simular a visão humana na interpretação das cenas do mundo real. Segundo~\citeonline{dawson2014}, a visão computacional é a análise automática de imagens e vídeos com o intuito de obter alguma informação ou dado do ambiente que cerca o sensor de visão. Assim, a visão computacional prove a capacidade de extrair informações de uma imagem que possa servir de base direta ou indireta em tomadas de decisões futuras, como por exemplo orientação de um robô com sensores visuais.

Segundo~\citeonline{forsyth2012}, a visão computacional tem uma grande variedade de aplicações, tanto antigas (por exemplo, robô móvel de navegação, inspeção industrial e inteligência militar) como novas (por exemplo, interação com computadores humanos, recuperação de imagens em bibliotecas digitais, análise de imagens médicas e a renderização realista de cenas em computação gráfica).

Um exemplo de aplicação de visão computacional seria o reconhecimento de caracteres em uma placa de um carro conforme demonstra a~\autoref{fig:id_placa} mostra e que através de uma série de processos na imagem chega-se a um resultado a ser avaliado (tomada de decisão). 

\begin{figure}[htb]
	\centering
	\caption{{\footnotesize Identificação dos Caracteres da Placa Veicular.}}   % nome da figura
	\label{fig:id_placa}
	\includegraphics[width=.7\textwidth]{1.pdf}
	
	%\legend{Fonte: \citeonline{Vestibulando_2016}.}   % outra opção de legenda da foto
	{\footnotesize Fonte: \citeonline{araujo2015}.}
\end{figure}

\section{\uppercase{Etapas do Processo}}

Para se chegar a um resultado satisfatório no reconhecimento de imagem, se faz necessário que a imagem passe por processos importantes iniciados pela aquisição da imagens (ou vídeos), pré-processamento, segmentação, seleção e extração de características e classificação, até chegar ao resultado. A descrição dos processos é mostrada na \autoref{fig:etapas} conforme a sequência de processamento. Estes passos serão melhor explanados ao longo deste capitulo.

\begin{figure}[htb]
	\centering
	\caption{{\footnotesize Etapas do processo para o reconhecimento de imagens.}}   % nome da figura
	\label{fig:etapas}
	\includegraphics[width=.7\textwidth]{2.pdf}
	
	%\legend{Fonte: \citeonline{Vestibulando_2016}.}   % outra opção de legenda da foto
	{\footnotesize Fonte: Próprio Autor, 2016 (Adaptado de~\cite{juliana2016}).}
\end{figure}

\subsection{\textbf{Pré-processamento}}

A imagem ou vídeo adquirido em meio as condições do processo de aquisição pode conter imperfeições. No intuito de utilizar essa imagem nas demais fases do processo. De acordo com~\cite{pedrini2017} que afirma:

\begin{citacao}
	
	A etapa de pré-processamento visa melhorar a qualidade da imagem por meio da aplicação de técnicas para atenuação de ruído, correção de contraste ou brilho e suavização de determinadas propriedades da imagem.
	
\end{citacao}
	 

\subsection{\textbf{Segmentação}}

Para~\citeonline{forsyth2012}, a ideia central de segmentação é coletar \emph{pixels} ou elementos de padrão em representações sumárias que enfatizam propriedades importantes, interessantes e distintivas. Assim, conforme~\citeonline{sobral2003} a segmentação tem por objetivo em regiões ou objetos, frequentemente ocasionando não uma imagem como resultado, mas um conjunto de regiões ou objetos.

Segundo~\citeonline{agarwal2016}, o processo de segmentação pode ser feito em duas abordagens distintas: por descontinuidade, que seria a segmentação de bordas e contornos ou; por similaridade que seria, binarização, crescimento de regiões, divisão e junção de regiões. A~\autoref{fig:proc_gradiente} ilustra um exemplo do processo de segmentação por descontinuidade.

\begin{figure}[htb]
	\centering
	\caption[\footnotesize Processo de Segmentação por Descontinuidade.]{\footnotesize Processo de Segmentação por Descontinuidade. a) Imagem de entrada; b) $G_y$ Componente do gradiente vertical; c) $G_x$ Componente do Gradiente horizontal; d) Resultado dos contornos ligados entre si.}   % nome da figura
	\label{fig:proc_gradiente}
	\begin{subfigure}{.4\textwidth}
		\centering
		\includegraphics[width=.7\linewidth]{3a.pdf}
		\caption{ }
	\end{subfigure}
	\begin{subfigure}{.4\textwidth}
		\centering
		\includegraphics[width=.7\linewidth]{3b.pdf}
		\caption{ }
	\end{subfigure}
	\\
	\begin{subfigure}{.4\textwidth}
		\centering
		\includegraphics[width=.7\linewidth]{3c.pdf}
		\caption{ }
	\end{subfigure}
	\begin{subfigure}{.4\textwidth}
		\centering
		\includegraphics[width=.7\linewidth]{3d.pdf}
		\caption{ }
	\end{subfigure}
	\\
	%\legend{Fonte: \citeonline{Vestibulando_2016}.}   % outra opção de legenda da foto
	{\footnotesize Fonte:~\citeonline{sobral2003}.}
\end{figure}

\subsection{\textbf{Seleção e Extração de Características}}

Seleção de características também é chamada seleção de variável ou seleção de atributo. É a seleção automática de atributos e seus dados que são mais relevantes para o problema de modelagem, é diferente da redução de dimensionalidade \cite{brownlee2014}. Os dois métodos procuram reduzir o número de atributos no conjunto de dados, mas um método de redução de dimensionalidade o faz criando novas combinações de atributos, onde como métodos de seleção de recursos incluem e excluem atributos presentes nos dados sem alterá-los \cite{brownlee2014}.

Existem dois métodos principais para reduzir a dimensionalidade: seleção de características e extração de características. Na seleção de características, tem-se o interesse de encontrar $k$ características das dimensões $d$ que dão mais informações e descartamos as outras dimensões $(d - k)$ \cite{alpaydin2014}.

Segundo \citeonline{alpaydin2014}, esses métodos podem ser supervisionados ou não supervisionados dependendo se eles usam ou não as informações de saída. 

\subsection{\textbf{Classificação ou Reconhecimento de Padrão}}

Reconhecimento de padrão de acordo com \citeonline{agarwal2016} é o processo de classificação de entrada de dados dentro de uma classe sendo essa classificação baseada em dois métodos aprendizagem.

Segundo \cite{bianchi2006}, há duas formas de conhecimento de padrão: classificação supervisionada; onde o padrão de entrada tem sua identificação Projetada para uma classe pré-definida ou classificação não supervisionada; onde o padrão é estabelecido por uma fronteira de classe desconhecida. 

Para \citeonline{duda2001}, uma das fases do reconhecimento padrão é a classificação de objetos, caracterizando e agrupando por categorias o objeto de interesse ao modelo, podendo ter um treinamento anterior ou não para a classificação.

Uma dificuldade no reconhecimento de padrão consiste na tarefa de categorização, onde as classes são definidas pelo projetista do sistema (classificação supervisionada) ou são ``aprendidas'' de acordo com a similaridade dos padrões (classificação não supervisionada) \cite{bianchi2006}.

Com o crescente avanço tecnológico, cresceu também o interesse em classificação de padrões, por permitir que aplicações computacionais possam auxiliar em diversas áreas do conhecimento. A \autoref{tab:padrao} mostra exemplos de aplicações com classes de padrões.

 \begin{table}[htb]
	\ABNTEXfontereduzida
	\caption[\footnotesize Exemplos de aplicações para o reconhecimento de padrões.]{\footnotesize Exemplos de aplicações para o reconhecimento de padrões.}
	\label{tab:padrao}
	\begin{tabular}{p{3cm}|p{3cm}|p{4cm}|p{4cm}}
		\hline
		\textbf{Área} & \textbf{Utilização} & \textbf{Entradas de Dados} & \textbf{Classes Padrão} \\
		\hline \hline
		\textbf{Bioinformática} & Análise de sequência	& DNA/Sequência de Proteína	& Tipos conhecidos de genes/padrões \\
		\hline
		\textbf{Mineração de dados} & Busca por padrões Significantes & Pontos em um espaço multidimensional & Compactar e separar grupos \\
		\hline
		\textbf{Classificação de documentos} & Busca na Internet & Documento texto & Categorias semânticas (negócios, entre outros) \\
		\hline
		\textbf{Análise de documento de imagem} & Máquina de leitura para cegos & Imagem de Documentos & Caracteres alfanuméricos, palavras \\
		\hline
		\textbf{Automação industrial} & Inspeção de placas de circuito impresso & Intensidade ou alcance de imagem & Natureza do produto (defeituosa ou não) \\ 
		\hline
		\textbf{Recuperação de base de dados multimídia} & Busca pela Internet & \textit{Video Clip} & Gêneros de vídeo (p.e. ação, diálogo, entre outros.) \\
		\hline
		\textbf{Reconhecimento biométrico} & Identificação pessoal & Face, íris, impressão digital & Usuários autorizados para controle de acesso \\
		\hline
		\textbf{Sensoriamento remoto} & Prognóstico da produção de colheita & Imagem multiespectral & Categorias de aproveitamento de terra, desenvolvimento de padrões de colheita \\
		\hline
		\textbf{Reconhecimento de voz} & Inquérito por telefone sem assistência de operador & Voz em forma de onda & Palavras faladas \\
		\hline \hline
	\end{tabular}
	\legend{\footnotesize Fonte: Próprio autor, 2017 (adaptado de \cite{bianchi2006}).}
\end{table} 

\section{\uppercase{Classificador}}

\subsection{\textbf{k-Nearest Neighbors (kNN)}}

Segundo~\citeonline{rosebrock2016}, kNN é de longe o mais simples algoritmo de classificação de imagem e aprendizado de máquina. A operação simplesmente compara um objeto a ser classificado com todos os objetos em uma série de treinamentos com rótulos de classes já conhecidas e tenta indicar o voto para qual classe atribuir o objeto \cite{solem2012}.

Para~\citeonline{luiza2005}, kNN é um classificador que tem como base de aprendizado à similaridade. O Conjunto de treinamento é formado por vetores $n$-dimensionais e cada elemento deste conjunto representa um ponto no espaço $n$-dimensional. Afim de obter a classe de um elemento que seja desvinculado ao conjunto de treinamento, o algoritmo busca por $K$ elementos do conjunto de treinamento tenham a menor distância. Ainda sobre as palavras diretas de \cite{luiza2005}: ``Estes $K$ elementos são chamados de $K$-vizinhos mais próximos (kNN). Verifica-se quais são as classes desses $K$ vizinhos e a classe mais frequente será atribuída à classe do elemento desconhecido.''

A seguir, apresenta-se as métricas comumente utilizadas no cálculo da distância entre dois pontos. O método utilizado com mais frequência é a distância Euclidiana \autoref{eq:euclid}. Outras métricas como as de Manhattan e de Minkowski, são apontadas nas \autoref{eq:manh} e \autoref{eq:mink}, respectivamente. O ultimo é uma generalização das duas distâncias anteriores.

Seja uma imagem $I$ de tamanho $X \times Y$ e $X=(x_1,x_2,x_3,\ldots,x_n)$ e $Y=(y_1,y_2,y_3,\ldots,y_n)$, dois conjuntos de pontos no plano $\Re^k$, onde $k$ é 2 se a imagem for bidimensional.

\begin{equation} \label{eq:euclid}
D(x,y) = \sqrt{(x_1-y_1)^2 + (x_2-y_2)^2 + \ldots + (x_n - y_n)^2}
\end{equation}

\begin{equation} \label{eq:manh}
D(x,y) = |x_1-y_1|+|x_2-y_2|+\ldots+|x_n-y_n|
\end{equation}

\begin{equation} \label{eq:mink}
D(x,y) = \sqrt[q]{|x_1-y_1|^q + |x_2-y_2|^q + \ldots + |x_n-y_n|^q} \rightarrow q \in \aleph
\end{equation}

Quando $q = 1$, a distância de Minkowski representa a distância de Manhattan e quando $q = 2$, a distância Euclidiana.

A distância Euclidiana ponderada também pode ser representada conforme~\citeonline{luiza2005} cita, se cada variável possuir um peso relativo à sua importância:

\begin{citacao}
	
	Pesos também podem ser aplicados às distâncias Manhattan e Minkowski. O kNN é um classificador que possui apenas um parâmetro livre (o número de K-vizinhos) que é controlado pelo usuário com o objetivo de obter uma melhor classificação. Este processo de classificação pode ser computacionalmente exaustivo se considerado um conjunto com muitos dados. Para determinadas aplicações, no entanto, o processo é bem aceitável.	

\end{citacao}

A \autoref{fig:knn_ex} mostra um exemplo de classificação pelo algoritmo kNN com as seguintes descrições:

\begin{itemize}
	\item Dois atributos;
	\item Três classes;
	\item Dois pontos desconhecidos (1 e 2); e
	\item Deseja-se classificar estes dois pontos através dos 7 vizinhos mais próximos.
\end{itemize}

\begin{figure}[htb]
	\centering
	\caption{{\footnotesize Classificação pelo método kNN.}}   % nome da figura
	\label{fig:knn_ex}
	\includegraphics[width=.7\textwidth]{4.pdf}
	
	%\legend{Fonte: \citeonline{Vestibulando_2016}.}   % outra opção de legenda da foto
	{\footnotesize Fonte:~\citeonline{luiza2005}.}
\end{figure}

\section{\uppercase{Framework OpenCV}}

OpenCV é um conjunto de bibliotecas \emph{Open-Source} de visão computacional e de aprendizagem de máquina \cite{pavalenko2017} que ajuda a cientistas e pesquisadores na área de processamento digital de imagens (PDI) a resolver problemas usando uma linguagem de programação livre e que possui uma comunidade enorme em termos de exemplos e aplicações educacionais e até mesmo, comerciais.

Segundo \citeonline{pavalenko2017}, o \emph{framework} possui hoje em dia mais de 2500 algoritmos clássicos e de estado da arte em visão computacional, PDI e aprendizado de máquina. Inicialmente o OpenCV começou como uma biblioteca desenvolvida para o Fortran pela Intel em 1999. Porém, esse sistema foi abandonado em meados de 2000 e, posteriormente, tido o suporte de colaboradores da \emph{Willow Garage}, que mais tarde se tornaria na OpenCV Foundation. O objetivo do software é diminuir o \emph{Gap} entre a visão computacional e programadores ao redor do mundo, principalmente quando se trata de aplicações comerciais e fechadas providas pela MathWorks e outras.

Ainda de acordo com \citeonline{pavalenko2017}, o software OpenCV foi escrito em C++, mas pode ter suas aplicações desenvolvidas nas linguagens C, Python, Java e MATLAB. Além de poder ser instalado em plataformas como Windows, Linux, Android e Mac OS. Hoje a mesma se encontra na versão 3.2 encontrada tanto no \href{http://opencv.org/releases.html}{site oficial} quanto no \href{https://github.com/opencv/opencv}{GitHub}. 

\section{\uppercase{System-on-Chip: Raspberry Pi}}

O Raspberry Pi é um microcomputador da família dos \textit{Single Board Computers} (SBC) do tamanho de um cartão de crédito que foi desenvolvido no Reino Unido pela Raspberry Pi Foundation \cite{pajankar2017}.

Ainda de acordo com \citeonline{pajankar2017}, objetivo por trás da criação da criação do Raspberry Pi foi de promover o ensino básico de ciência da computação em escolas de países em desenvolvimento fornecendo uma plataforma de computação de baixo custo. Por isso, a plataforma foi escolhida, visto que o ultimo modelo lançado (em fevereiro de 2016) suporta tanto sistemas operacionais Windows e Linux. Além de ser o sistema de \emph{System-on-Chip} (SoC) de alta performance mais utilizado no mundo.

Na \autoref{tab:rasp_specs}, mostra as especificações do modelo do Raspberry Pi mais recente (que no caso é o modelo 3B) e que será utilizado neste trabalho a fim desenvolver o projeto proposto.

\begin{table}[htb]
	\ABNTEXfontereduzida
	\caption[\footnotesize Especificações Técnicas do Raspberry Pi 3 Model B.]{\footnotesize Especificações técnicas do Raspberry Pi 3 Model B.}
	\label{tab:rasp_specs}
	\centering
	\begin{tabular}{c|p{11cm}}
		\hline
		\textbf{Item} & \textbf{Descrição} \\
		\hline \hline
		\textbf{Processador} & 1.2 GHz Quad-Core ARM Cortex-A53 de 64 bits\\
		\hline
		\textbf{Chipset} & Broadcom BCM2387 \\
		\hline
		\textbf{Wifi} & 802.11 b/g/n Wireless LAN \\
		\hline
		\textbf{LAN} & 10/100 BaseT Ethernet  \\
		\hline
		\textbf{Bluetooth} & Bluetooth 4.1 (Bluetooth Clássico e BLE) \\ 
		\hline
		\textbf{RAM} & 1GB RAM DDR3 \\
		\hline
		\textbf{Portas} & HDMI, P2 para áudio estéreo, Vídeo Composto, Conexão CSI para câmera, Conexão DSI para display de toque e 4 USB \\
		\hline
		\textbf{Fonte} & Conexão de fonte bivolt de 5VDC microUSB \\
		\hline
		\textbf{Cartão de Memoria} & Suporte a SD Card de até 256GB \\
		\hline \hline
	\end{tabular}
	\legend{\footnotesize Fonte: Próprio autor, 2017 (adaptado de \cite{pajankar2017}).}
\end{table} 

A \autoref{fig:rasp_specs} demonstra com mais detalhes a placa do Raspberry Pi 3 com a descrição de seus elementos.

\begin{figure}[htb]
	\centering
	\caption{{\footnotesize Detalhes do Raspberry Pi 3 Model B.}}   % nome da figura
	\label{fig:rasp_specs}
	\includegraphics[width=.8\textwidth]{5.pdf}
	
	%\legend{Fonte: \citeonline{Vestibulando_2016}.}   % outra opção de legenda da foto
	{\footnotesize Fonte:~\citeonline{element2016}.}
\end{figure}

\section{\uppercase{Considerações Finais}}

O escopo deste capitulo foi apresentar de forma sucinta sobre visão computacional, as etapas do processo de visão computacional até a tomada do resultado, discorrer um pouco sobre a biblioteca OpenCV a ser utilizada nesse trabalho, sobre o microcomputador Raspberry Pi que terá o seu papel como o processador central das tarefas que virão na proposta objetiva desse trabalho.

Este capitulo também servirá de base compreensiva para capítulos posteriores que apresentarão nomenclaturas relacionadas proposição da melhor maneira de resolver os problemas que possam surgir ao longo desse trabalho.
