\chapter{\textbf{\uppercase{Resultados}}}\label{cap_5}

Neste capítulo serão apresentados os resultados obtidos durante o transcorrer do projeto com base nas etapas do método proposto. 

Com o sistema em funcionamento, a permissão da aquisição da imagem se deu por meio de um sensor ultrassônico que mediu a proximidade das placas numa de média entre 50 a 60 cm de distância. Quando o sensor interpretava a média de distância estipulada, a câmera era acionada e realizava a aquisição da imagem. A mesma é enviada para o próximo passo, processamento digital da imagem, que por sua vez passava a imagem processada para as demais etapas do processo de reconhecimento da placa e dos caracteres.

Para realizar as simulações das placas de carros, foram utilizadas placas impressas em um papel cartão com as características da placa de um veiculo com placa brasileira (formato, fonte e tamanho), conforme a \autoref{fig:placa_simul}.

\begin{figure}[htb]
	\centering
	\caption{{\footnotesize Placa usada como exemplo para simular uma placa real.}}   % nome da figura
	\label{fig:placa_simul}
	\includegraphics[width=.9\textwidth]{25.pdf}
	
	%\legend{Fonte: \citeonline{Vestibulando_2016}.}   % outra opção de legenda da foto
	{\footnotesize Fonte: Próprio autor, 2017.}
\end{figure}

Para alcançar os resultados no sistema de reconhecimento, cada placa passou por um mínimo de três vezes na execução do sistema para que não houvesse dúvidas na precisão do reconhecimento. A \autoref{fig:placa_capture} mostra o momento em que a \textit{webcam} está capturando a imagem da placa sendo acionada pelo sinal que mede a distância determinada no algoritmo através do sensor ultrassônico.

\begin{figure}[htb]
	\centering
	\caption{{\footnotesize Momento da captura da imagem da placa (simulação).}}   % nome da figura
	\label{fig:placa_capture}
	\includegraphics[width=.5\textwidth]{26.pdf}
	
	%\legend{Fonte: \citeonline{Vestibulando_2016}.}   % outra opção de legenda da foto
	{\footnotesize Fonte: Próprio autor, 2017.}
\end{figure}

O tempo de execução do sistema foi de aproximadamente 10 segundos entre a aquisição da imagem a abertura do portão. O principal fator nesta demora é o acionamento do sensor da câmera e a captura da imagem (correspondendo a 63\% do tempo de operação). Em outro momento, pode-se ver através da \autoref{fig:placa_hardware} os detalhes da \textit{webcam}, do sensor ultrassônico, do Raspberry e o circuito que interliga o hardware do sistema.

\begin{figure}[htb]
	\centering
	\caption{{\footnotesize Detalhes do hardware que compõe o sistema.}}   % nome da figura
	\label{fig:placa_hardware}
	\includegraphics[width=.5\textwidth]{27.pdf}
	
	%\legend{Fonte: \citeonline{Vestibulando_2016}.}   % outra opção de legenda da foto
	{\footnotesize Fonte: Próprio autor, 2017.}
\end{figure}

A \autoref{tab:results} mostra em uma coluna as quantidades de caracteres alfanuméricos reconhecidos pelo sistema conforme a distância determinada que foi menor ou igual a 100 cm tendo os melhores resultados na média de 50 a 60 cm de distância da câmera. Além de mostrar a taxa de acerto em termos de quantos caracteres foram corretamente identificados da placa. 

No teste, foram utilizadas 26 placas diferentes nas fontes Arial e \textit{Mandatory}. Essa pontuação serve como base para a comprovação da eficiência do código de abertura do portão. E lembrando também que as condições de iluminação estavam favoráveis, devido a iluminação artificial adicionada a cena.

\begin{table}[htb]
	\ABNTEXfontereduzida
	\caption[\footnotesize Resultado do processo de reconhecimento de caracteres.]{\footnotesize Resultado do processo de reconhecimento de caracteres.}
	\label{tab:results}
	\begin{tabular}{c|c|c|c|c|c}
		\hline
		\textbf{\#} & \textbf{Placa} & \textbf{Distância (cm)} & \textbf{Caracteres Reconhecidos} & \textbf{Qtde. de Acertos} & \textbf{\%} \\
		\hline \hline
		1 & RCI-1973 & 56,75 & RCI1973 & 7 & 100\% \\
		\hline
		2 & CSC-2013 & 57,61 & CSC2013 & 7 & 100\% \\
		\hline
		3 & HQW-5678 & 57,66 & 15678 & 4 & 57\% \\
		\hline
		4 & BCD-3456 & 57,38 & BCD3456 & 7 & 100\% \\
		\hline
		5 & ABC-1234 & 46,74 & 4BC1234 & 6 & 86\% \\
		\hline
		6 & CEL-8142 & 59,01 & CEL8142 & 7 & 100\% \\
		\hline
		7 & JBS-2017 & 60,45 & JBS2017 & 7 & 100\% \\
		\hline
		8 & GAY-2424 & 58,54 & GAY2424 & 7 & 100\% \\
		\hline
		9 & SAP-5470 & 59,94 & SAP5470 & 7 & 100\% \\
		\hline
		10 & RIO-2016 & 60,36 & R2016 & 5 & 71\% \\
		\hline
		11 & ARG-0316 & 59,98 & ARG0316 & 7 & 100\% \\
		\hline
		12 & VEN-8374 & 59,13 & VEN8374 & 7 & 100\% \\
		\hline
		13 & EUA-6574 & 69,63 & EU6574 & 6 & 86\% \\
		\hline
		14 & BRA-1980 & 58,76 & BRA1980 & 7 & 100\% \\
		\hline
		15 & PTZ-1437 & 59,64 & PTZ1437 & 7 & 100\% \\
		\hline
		16 & MAN-9128 & 59,52 & JAN9128 & 6 & 86\% \\
		\hline
		17 & DIR-6456 & 59,22 & DR6456 & 6 & 86\% \\
		\hline
		18 & OAC-7473 & 58,79 & OAC7473 & 7 & 100\% \\
		\hline
		19 & JWY-3080 & 58,46 & JY3080 & 6 & 86\% \\
		\hline
		20 & NOW-2371 & 58,54 & 2371 & 4 & 57\% \\
		\hline
		21 & CMN-7472 & 58,69 & CMN7472 & 7 & 100\% \\
		\hline
		22 & NOP-3034 & 58,37 & NOP3034 & 7 & 100\% \\
		\hline
		23 & NOR-0316 & 71,85 & NOR0316 & 7 & 100\% \\
		\hline
		24 & BEE-4R22 & 65,76 & BEE4R22 & 7 & 100\% \\
		\hline
		25 & TWX-2552 & 58,03 & TX2552 & 6 & 86\% \\
		\hline
		26 & PHL-7452 & 57,94 & PHL7452 & 7 & 100\% \\
		\hline
		\multicolumn{5}{r|}{Total de confiabilidade (caracteres)} & 92,34\% \\
		\multicolumn{5}{r|}{Total de confiabilidade (placas)} & 65,38\% \\  
		\hline \hline
	\end{tabular}
	\legend{\footnotesize Fonte: Próprio autor, 2017.}
\end{table} 

Para que se possa demonstrar o resultado do reconhecimento dos caracteres da placa. A \autoref{fig:resullt_screen} mostra como aparece o resultado final para quem supervisiona o sistema em funcionamento.

\begin{figure}[htb]
	\centering
	\caption{{\footnotesize Resultado do reconhecimento no sistema.}}   % nome da figura
	\label{fig:resullt_screen}
	\includegraphics[width=.7\textwidth]{28.pdf}
	
	%\legend{Fonte: \citeonline{Vestibulando_2016}.}   % outra opção de legenda da foto
	{\footnotesize Fonte: Próprio autor, 2017.}
\end{figure}

Com base nos custos do projeto foi possível realizar uma comparação de valores com um outro sistema prontos no mercado e nacional da Intelbrás OCR LPR-ITS, visto na \autoref{fig:lpr}. A \autoref{tab:compar} demonstra o comparativo de preços do mesmo sistema em sites/estabelecimentos diferentes no Brasil.

\begin{figure}[htb]
	\centering
	\caption{{\footnotesize Sistema de reconhecimento de placas veiculares da Intelbrás.}}   % nome da figura
	\label{fig:lpr}
	\includegraphics[width=.6\textwidth]{29.pdf}
	
	%\legend{Fonte: \citeonline{Vestibulando_2016}.}   % outra opção de legenda da foto
	{\footnotesize Fonte: Google Imagens, 2017.}
\end{figure}

\begin{table}[htb]
	\ABNTEXfontereduzida
	\caption[\footnotesize Comparação de valores.]{\footnotesize Comparação de valores de produtos disponíveis no mercado atual.}
	\label{tab:compar}
	\centering
	\begin{tabular}{c|c|c|c}
		\hline
		\textbf{Site/Estabelecimento} & \textbf{Produto} & \textbf{Fabricante} & \textbf{Preço encontrado} \\
		\hline \hline
		Casas Bahia & \multirow{3}{*}{\textbf{OCR LPR-ITS}} & \multirow{3}{*}{\textbf{Intelbrás}} & R\$ 26.145,00 \\
 		\cline{1-1} 
 		\cline{4-4}
 		MercadoLivre & & & R\$ 28.884,00 \\
 		\cline{1-1} 
 		\cline{4-4}
 		FX Biometria & & & R\$ 24.900,00 \\
		\hline \hline
	\end{tabular}
	\legend{\footnotesize Fonte: Próprio autor, 2017.}
\end{table} 

De acordo com a \autoref{tab:compar}, tomou-se como base o valor mais baixo que foi de R\$ 24.900,00. A economia do sistema desenvolvido ficou em torno de 97\% mais baixo que em relação ao preço de mercado na FX Biometria. Com essa comparação pode se ter uma ideia de que o projeto torna-se economicamente viável em uma possível implantação. 