%% abtex2-modelo-include-comandos.tex, v-1.9.2 laurocesar
%% Copyright 2012-2014 by abnTeX2 group at http://abntex2.googlecode.com/ 
%%
%% This work may be distributed and/or modified under the
%% conditions of the LaTeX Project Public License, either version 1.3
%% of this license or (at your option) any later version.
%% The latest version of this license is in
%%   http://www.latex-project.org/lppl.txt
%% and version 1.3 or later is part of all distributions of LaTeX
%% version 2005/12/01 or later.
%%
%% This work has the LPPL maintenance status `maintained'.
%% 
%% The Current Maintainer of this work is the abnTeX2 team, led
%% by Lauro César Araujo. Further information are available on 
%% http://abntex2.googlecode.com/
%%
%% This work consists of the files abntex2-modelo-include-comandos.tex
%% and abntex2-modelo-img-marca.pdf
%%

% ---
% Este capítulo, utilizado por diferentes exemplos do abnTeX2, ilustra o uso de
% comandos do abnTeX2 e de LaTeX.
% ---
 
\chapter{\textbf{\uppercase{Contextualização}}}\label{cap_1}

Devido ao crescimento populacional e suas necessidades, as frotas de veículos também estão aumentando exponencialmente, isso reflete diretamente na necessidade de se ter um maior acesso ao controle de automóveis. Com a crescente demanda pelo aumento de segurança em locais fechados e um maior controle no acesso de seus moradores, torna-se hoje em dia imprescindível que a segurança eletrônica seja automatizada, rápida e eficaz na identificação de seus moradores. Além da observação e verificação humana, o sistema de reconhecimento de veículos através da identificação da sua placa pode aumentar a segurança. Por isso, é de suma importância que sistemas de reconhecimento automático de placas de carro sejam implementados em ambientes reservados e particulares e com um melhor controle de acesso para seus moradores, onde a intervenção humana seja minimizada ou quase não necessária nesse contexto, apenas a supervisão. 

A tecnologia atual tornou-se barata, competitiva e compacta, então torna-se fácil implementar sistemas de certa robustez em dispositivos com essas características, como é o caso do microcomputador Raspberry Pi utilizado nesse trabalho. Com sistemas e Hardwares \textit{Open Source} se popularizando, torna-se barato implementar sistemas dessa natureza aumentando o nível de segurança e controle de veículos que trafegam dentro de condomínios.

% ---
\section{\uppercase{Justificativa}}
% ---

O reconhecimento automático de placa de matricula de carros é uma tarefa importante que visa automatizar o controle de acesso em portões eletrônicos. Esse tipo de aplicação já é muito utilizado em muitos setores da sociedade, segundo \citeonline{babu2016}: ``A detecção de placas de matrícula pode ser usada em muitas aplicações, tais como controle de tráfego, controle de velocidade, identificação dos carros roubados, portões de pedágio, aplicação de segurança, etc.''.

Então, com base neste pretexto, o autor buscou um método simples e automático para tentar aumentar a segurança de sistemas que usam cancelas para acesso a determinados ambientes de estacionamento veicular. Como a aplicação utilizará basicamente uma câmera e algoritmos, o custo e a replicação do experimento são de fácil e livre acesso ao todos. Apenas algumas peças de hardware devem ser adicionadas ao projeto final em ordem de completar o processo como um produto completo para qualquer situação (dentro do escopo de acesso veicular).

% ---
\section{\uppercase{Motivação}}
% ---

Este trabalho foi motivado pela espera na identificação de um veículo e seu condutor de forma manual em duas portarias num determinado ambiente reservado, onde se faz a necessidade da criação de aplicações automatizadas para o controle de acesso a lugares reservados de modo rápido, preciso e quase sem a necessidade da intervenção humana através de uma forma eficiente de se obter informação ao acesso de veículos. Visto que, é comum nos dias de hoje haverem acessos indevidos a condomínios e até mesmo a residências urbanas por estranhos sem autorização e ladrões. O que intriga o autor, é que tecnologias com esta citada no trabalho, já são bem comuns em cidades do Brasil a anos e na sua cidade, Manaus-AM, é raro ver esse tipo de adicional a segurança, mesmo em condomínios e prédios relativamente novos.

% ---
\section{\uppercase{Problemática}}
% ---

A demora na identificação de um veículo e seu condutor feito de forma manual e acessos indevidos em ambientes fechados e um sistema de controle eficiente, são problemas que o autor tenta buscar a solução através do um desenvolvimento possível de um sistema de reconhecimento automático de placas veiculares, que é uma tecnologia bastante usada hoje em dia pela computação, mas pouco explorada por acadêmicos de engenharia elétrica. Por ser, geralmente, de código aberto e gratuito, os algoritmos que são utilizados na Visão Computacional podem solucionar o problema com eficácia, na visão do autor.  

% ---
\section{\uppercase{Hipóteses}}
% ---

\begin{itemize}
	\item A utilização de uma \textit{Webcam} de baixo custo poderá afetar na resolução da imagem adquirida. Contudo, é possível adicionar algumas etapas de processamento digital de imagens com uso de filtros de correção de cor e de iluminação para maximizar a qualidade da imagem;
	\item Poderá haver erros na identificação dos caracteres nas placas dos carros, devido ao fato que nenhum método é perfeito. O algoritmo de treinamento de identificação dos caracteres pode não reconhecer corretamente ou haver alguma distorção na imagem que impossibilite o reconhecimento de algum caractere. Ainda lembrando que, dependendo da fonte utilizada na placa, o treinamento de caracteres pode perder muito a taxa de acerto. A fonte deve ser levada em conta aqui;
	\item Pode ser possível que, mesmo com todas as soluções apontadas, a câmera acabe não reconhecendo a placa devido alguns problemas na cena. Por isso, a câmera ficará colocada em posição bem favorável para captura correta da imagem da placa. Talvez, seja necessário ajustar o ambiente para melhorar a captura, como iluminação artificial e posicionamento da câmera para evitar oclusões.
\end{itemize}

% ---
\section{\uppercase{Objetivos}}
% ---
Serão apresentados os seguintes objetivos deste trabalho a partir das análises realizadas inicialmente para essas abordagens.

\subsection{\textbf{Objetivo Geral}}

Desenvolver um sistema de reconhecimento automático da placas de matricula veicular para aumentar a segurança e o controle de acesso de veículos em ambientes reservados. 

\subsection{\textbf{Objetivos Específicos}}

\begin{itemize}
	\item [\textbf{a.}] Aumentar a performance do algoritmo de detecção de caracteres em ordem de diminuir os erros;
	\item [\textbf{b.}] Desenvolver um protótipo de um portão eletrônico para simular o funcionamento do sistema e construir o algoritmo de acionamento do mesmo;
	\item [\textbf{c.}] Realizar a comparação de custo de construção do protótipo do sistema com outros sistemas prontos no mercado para verificar se o valor de acesso se mantem baixo e de fácil acesso.
\end{itemize}

\section{\uppercase{Organização do Trabalho}}

Este trabalho está estruturado em ordem sequencial com uma breve descrição dos conteúdos dos capítulos:

\begin{itemize}
	\item \textbf{Fundamentação Teórica:} é o capitulo 2 deste trabalho. Apresenta conceitos e definições de assuntos relacionados a visão computacional e suas etapas. São estudadas as teorias relacionados ao processo de reconhecimento da imagem, das ferramentas (aplicativos e as bibliotecas utilizadas do OpenCV) e do hardware empregado neste trabalho;
	\item \textbf{Trabalhos Correlatos:} é o capitulo 3 deste trabalho. Mostra vários trabalhos que foram abordados na área de visão computacional que serviram de embasamento para a busca de uma solução no reconhecimento automático de placas veiculares utilizando reconhecimento de padrões;
	\item \textbf{Método Proposto:} é o capitulo 4 deste trabalho. Aqui é abordado a forma de como chegar a solução do objetivo geral proposto, apresentando os métodos para alcançar a resolução desejada; 
	\item \textbf{Resultados:} é o capitulo 5 deste trabalho. Apresenta os resultados obtidos após a conclusão do objetivo geral e do específicos que foram concluídos, falando também das suas possíveis limitações;
	\item \textbf{Conclusão e projetos futuros:} finaliza o trabalho trazendo abordagens sobre o sucesso dos resultados obtidos e possíveis falhas nos objetivos específicos que não foram alcançados, sugerindo uma melhoria no método proposto e trabalhos futuros que venham agregar alguma possível melhoria ao projeto.
\end{itemize}
