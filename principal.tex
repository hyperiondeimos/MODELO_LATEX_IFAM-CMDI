% ----------------------------------------------------------
% VERSÃO ORIGINAL
% ----------------------------------------------------------
% The Current Maintainer of this work is the abnTeX2 team, led
% by Lauro César Araujo. Further information are available on 
% http://abntex2.googlecode.com/

% VERSÃO 0.1: 15 de outubro de 2017
% ----------------------------------------------------------
% Versão do modelo ABNTeX modificado para o Instituto Federal do Amazonas-CMDI
% Mantido por Anderson Gadelha Fontoura
% e-mail: anderson.fontoura@uninorte.com.br

\documentclass[
% -- opções da classe memoir --
12pt,				% tamanho da fonte
openright,			% capítulos começam em pág ímpar (insere página vazia caso preciso)
oneside,			% para impressão em verso e anverso coloque twoside
a4paper,			% tamanho do papel. 
% -- opções da classe abntex2 --
%chapter=TITLE,		% títulos de capítulos convertidos em letras maiúsculas
%section=TITLE,		% títulos de seções convertidos em letras maiúsculas
%subsection=TITLE,	% títulos de subseções convertidos em letras maiúsculas
%subsubsection=TITLE,% títulos de subsubseções convertidos em letras maiúsculas
% -- opções do pacote babel --
english,			% idioma adicional para hifenização
french,				% idioma adicional para hifenização
spanish,			% idioma adicional para hifenização
brazil				% o último idioma é o principal do documento
]{abntex2}


% ----------------------------------------------------------
% PACOTES
% ----------------------------------------------------------
% ----------------------------------------------------------
% PACOTES BÁSICOS
% ----------------------------------------------------------
\usepackage{lmodern}            % Usa a fonte Latin Modern          
\usepackage[T1]{fontenc}        % Selecao de codigos de fonte.
\usepackage[utf8]{inputenc}     % Codificacao do documento (conversão automática dos acentos)
\usepackage{lastpage}           % Usado pela Ficha catalográfica
\usepackage{indentfirst}        % Indenta o primeiro parágrafo de cada seção.
\usepackage{color}      		% Controle das cores
\usepackage{graphicx}           % Inclusão de gráficos
\usepackage{microtype}          % para melhorias de justificação
\usepackage{algorithm,algorithmic}          % Inserir código de linguagem de programação
\usepackage{float,array,multicol,multirow,booktabs,hhline,colortbl}		% pacotes para ajuda em formatação de tabelas
\usepackage{subcaption}			% para colocar multiplas imagens uma ao lado da outra
\usepackage[table]{xcolor}		% para dar stripes nas tabelas
\usepackage{amsmath}            % Pacote usado para aumentar as opções de simbolos em equações
\usepackage{enumitem}			% Usado para enumerar e melhorar as listas
\usepackage{listingsutf8}		% Serve para colocar códigos fonte com caracteres em UTF8, necessário em alguns tipos de scripts
\usepackage{textcomp}			% Serve para colocar simbolos especiais
\usepackage{gensymb}			% Serve para colocar simbolos especiais no modo matemático
%\usepackage[none]{hyphenat} 	% Serve para eliminar a hifenização no documento, que é proibido segundo a noram NBR 14724/11 
% descomente a linha acima APENAS se não houver problemas com texto passando a margem

% Pacotes de citações
\usepackage[brazilian,hyperpageref]{backref}     % Paginas com as citações na bibl
\usepackage[alf]{abntex2cite}   % Citações padrão ABNT

% CONFIGURAÇÕES DE PACOTES
% Configurações do pacote backref
% Usado sem a opção hyperpageref de backref
\renewcommand{\backrefpagesname}{Citado na(s) página(s):~}
% Texto padrão antes do número das páginas
\renewcommand{\backref}{}
% Define os textos da citação
\renewcommand*{\backrefalt}[4]{
    \ifcase #1 %
        Nenhuma citação no texto.%
    \or
        Citado na página #2.%
    \else
        Citado #1 vezes nas páginas #2.%
    \fi}%

% Pacotes adicionais, usados apenas no âmbito do Modelo Canônico do abnteX2
\usepackage{lipsum}             % para geração de dummy text

% TODO inserir seus pacotes aqui



% ESTILO
\usepackage{ifam-abntex2}

% ----------------------------------------------------------
% CAPA E FOLHA DE ROSTO
% ----------------------------------------------------------
% ----------------------------------------------------------
% CAPA E FOLHA DE ROSTO
% ----------------------------------------------------------

% AUTOR
\newcommand{\tccautor}{Fulano de Tal}

\titulo{\uppercase{Titulo Pincipal}}
\autor{\tccautor}
\local{Manaus, AM}
\data{2017}
\orientador{Prof. M.Sc. Anderson Gadelha Fontoura}
%\coorientador{Dr. Prof. Astucio Gilmar}
\instituicao{%
  Instituto Federal do Amazonas - IFAM
  \par
  Campus Manaus Distrito-Industrial - CMDI
  \par
  Curso de Engenharia de Controle e Automação}
%\tipotrabalho{Tese (Doutorado)}
\tipotrabalho{Trabalho de Conclusão de Curso (Monografia)}
% O preambulo deve conter o tipo do trabalho, o objetivo, 
% o nome da instituição e a área de concentração 
\preambulo{Trabalho de Conclusão de Curso submetido à Coordenação do Curso de Engenharia de Controle e Automação do Instituto Federal do Amazonas (IFAM-CMDI), como requisito parcial para obtenção do Título de Bacharel em Engenharia de Controle e Automação.}



% ----------------------------------------------------------
% CONFIGURAÇÕES
% ----------------------------------------------------------

% Configurações de aparência do PDF final

% alterando o aspecto da cor azul
\definecolor{blue}{RGB}{41,5,195}

% informações do PDF
\makeatletter
\hypersetup{
	%pagebackref=true,
	pdftitle={\@title}, 
	pdfauthor={\@author},
	pdfsubject={\imprimirpreambulo},
	pdfcreator={LaTeX with abnTeX2},
	pdfkeywords={abnt}{latex}{abntex}{abntex2}{trabalho acadêmico}, 
	colorlinks=true,            % false: links em boxs; true: links coloridos
	linkcolor=black,             % cor dos links de citação interna (\autoref)
	citecolor=black,             % cor dos links de bibliografia
	filecolor=black,          % cor dos links dos arquivos
	urlcolor=black,
	bookmarksdepth=4
}
\makeatother

% CONFIGS DOS TÍTULOS

% Configurações dos títulos dos capítulos e seções
%\renewcommand{\ABNTEXchapterfont}
\renewcommand{\ABNTEXchapterfontsize}{\normalsize}
%\renewcommand{\ABNTEXsectionfont}
\renewcommand{\ABNTEXsectionfontsize}{\normalsize}
%\renewcommand{\ABNTEXsubsectionfont}
\renewcommand{\ABNTEXsubsectionfontsize}{\normalsize}
%\renewcommand{\ABNTEXsubsubsectionfont}
\renewcommand{\ABNTEXsubsubsectionfontsize}{\normalsize}
%\renewcommand{\ABNTEXsubsubsubsectionfont}
\renewcommand{\ABNTEXsubsubsubsectionfontsize}{\normalsize}

% CONFIGS DE INDENTAÇÃO

% Espaçamentos entre linhas e parágrafos 
% O tamanho do parágrafo é dado por:
\setlength{\parindent}{1.25cm}

% Controle do espaçamento entre um parágrafo e outro:
\setlength{\parskip}{0.2cm}  % tente também \onelineskip

% compila o indice
\makeindex

% DIRETÓRIO PADRÃO DAS IMAGENS

\graphicspath{{imagens/}}       % informa ao programa qual é a pasta padrão das figuras

% ----------------------------------------------------------
% INÍCIO DOCUMENTO
% ----------------------------------------------------------
\begin{document}
	
	% Retira espaço extra obsoleto entre as frases.
	\frenchspacing
	
	
	% Verifica hifenização (como as palavras devem ser separadas por - )
	\hyphenation{ARDUINO plan-ta es-ta-ção} 
	
	% ----------------------------------------------------------
	% ELEMENTOS PRÉ-TEXTUAIS
	% ----------------------------------------------------------
	% \pretextual
	
	% Capa
	\imprimircapa
	
	% Folha de rosto
	% (o * indica que haverá a ficha bibliográfica)
	\imprimirfolhaderosto*
	
	
	% Descomente apenas na versão final a ficha catalografica e a folha de aprovação
	% Só descomente a "errata" se for necessário
	% ----------------------------------------------------------
% FICHA BIBLIOGRAFICA
% ----------------------------------------------------------

% Isto é um exemplo de Ficha Catalográfica, ou ``Dados internacionais de
% catalogação-na-publicação''. Você pode utilizar este modelo como referência. 
% Porém, provavelmente a biblioteca da sua universidade lhe fornecerá um PDF
% com a ficha catalográfica definitiva após a defesa do trabalho. Quando estiver
% com o documento, salve-o como PDF no diretório do seu projeto e substitua todo
% o conteúdo de implementação deste arquivo pelo comando abaixo:
%
% \begin{fichacatalografica}
%     \includepdf{fig_ficha_catalografica.pdf}
% \end{fichacatalografica}
\begin{fichacatalografica}
	\vspace*{\fill}					% Posição vertical
	\hrule							% Linha horizontal
	\begin{flushleft}
		R696m
	\end{flushleft}
	\begin{center}					% Minipage Centralizado
	\begin{minipage}[c]{12.5cm}		% Largura
	
	\imprimirautor
	
	\hspace{0.5cm} \imprimirtitulo  / \imprimirautor. --
	\imprimirlocal, \imprimirdata-
	
	\hspace{0.5cm} \pageref{LastPage} p. : il. (algumas color.) ; 30 cm.\\
	
	\hspace{0.5cm} \imprimirorientadorRotulo~\imprimirorientador\\
	
	\hspace{0.5cm}
	\parbox[t]{\textwidth}{\imprimirtipotrabalho~--~\imprimirinstituicao,
	\imprimirdata.}\\
	
	\hspace{0.5cm}
		1. Reconhecimento de Caracteres.
		2. Reconhecimento de Padrões.
		I. Anderson Gadelha Fontoura.
		II. Centro Universitário do Norte - UNINORTE.
		III. Curso de Engenharia Elétrica.
		IV. Melhoria de Segurança em Portões Eletrônicos Utilizando Reconhecimento de Padrões.\\ 			
	
	\hspace{8.75cm} CDU 621.319\\
	
	\end{minipage}
	\end{center}
	\hrule
\end{fichacatalografica}

	%% ----------------------------------------------------------
% INSERIR ERRATA
% ----------------------------------------------------------
\begin{errata}
Elemento opcional da \citeonline[4.2.1.2]{NBR14724:2011}. Exemplo:

\vspace{\onelineskip}

FERRIGNO, C. R. A. \textbf{Tratamento de neoplasias ósseas apendiculares com
reimplantação de enxerto ósseo autólogo autoclavado associado ao plasma
rico em plaquetas}: estudo crítico na cirurgia de preservação de membro em
cães. 2011. 128 f. Tese (Livre-Docência) - Faculdade de Medicina Veterinária e
Zootecnia, Universidade de São Paulo, São Paulo, 2011.

\begin{table}[htb]
\center
\footnotesize
\begin{tabular}{|p{1.4cm}|p{1cm}|p{3cm}|p{3cm}|}
  \hline
   \textbf{Folha} & \textbf{Linha}  & \textbf{Onde se lê}  & \textbf{Leia-se}  \\
    \hline
    1 & 10 & auto-conclavo & autoconclavo\\
   \hline
\end{tabular}
\end{table}

\end{errata}


	% ----------------------------------------------------------
% INSERIR FOLHA DE APROVAÇÃO
% ----------------------------------------------------------
% Isto é um exemplo de Folha de aprovação, elemento obrigatório da NBR
% 14724/2011 (seção 4.2.1.3). Você pode utilizar este modelo até a aprovação
% do trabalho. Após isso, substitua todo o conteúdo deste arquivo por uma
% imagem da página assinada pela banca com o comando abaixo:
%
% \includepdf{folhadeaprovacao_final.pdf}
%
\begin{folhadeaprovacao}

  \begin{center}
    {\ABNTEXchapterfont\bfseries\large\imprimirautor}

    \vspace*{\fill}\vspace*{\fill}
    \begin{center}
      \ABNTEXchapterfont\bfseries\Large\imprimirtitulo
    \end{center}
    \vspace*{\fill}
    
    \hspace{.45\textwidth}
    \begin{minipage}{\textwidth}
        \imprimirpreambulo       
    \end{minipage}%
    %\vspace*{\fill}
    \vspace{1cm}
    Trabalho aprovado em: 27 de junho de 2017 \\
    \imprimirlocal
    \\
    \vspace{10mm}
    \uppercase{\textbf{banca examinadora}} 
   \end{center}     

   \assinatura{\textbf{\imprimirorientador \hspace{2mm}(Presidente)} \\ UNINORTE} 
   \assinatura{\textbf{Prof. MSc. Francisco da Silva Coelho (Membro)} \\ UNINORTE}
   \assinatura{\textbf{Prof. MSc. Weverson dos Santos Cirino (Membro)} \\ UNINORTE}
   %\assinatura{\textbf{Professor} \\ Convidado 3}
   %\assinatura{\textbf{Professor} \\ Convidado 4}
      
%   \begin{center}
%    \vspace*{0.5cm}
%    {\large\imprimirlocal}
%    \par
%    {\large\imprimirdata}
%    \vspace*{1cm}
%  \end{center}
  
\end{folhadeaprovacao}


	% ----------------------------------------------------------
% DEDICATÓRIA
% ----------------------------------------------------------
\begin{dedicatoria}
   \vspace*{\fill}
   \centering
   \noindent
   \textit{Dedico este trabalho a minha mãe, minha esposa Geysy Almedia, meus filhos e a toda a minha família que, com muito esforço e carinho, não mediram esforços para que pudesse chegar até esta etapa de minha vida} \vspace*{\fill}
\end{dedicatoria}


	% ----------------------------------------------------------
% AGRADECIMENTOS
% ----------------------------------------------------------
\begin{agradecimentos}

Agradeço em primeiro lugar a Deus por ter me iluminado e permitido que até aqui eu chegasse. 

Agradeço ao corpo Docente dessa universidade. 

Agradeço ao meu orientador, professor Anderson Gadelha, pelo suporte no pouco tempo que lhe coube, pelas correções e incentivos. 

Agradeço a todos que direta ou indiretamente participaram da minha caminhada acadêmica. 

O meu muito obrigado. 

\end{agradecimentos}


	% ----------------------------------------------------------
% EPÍGRAFE
% ----------------------------------------------------------
\begin{epigrafe}
    \vspace*{\fill}
	\begin{flushright}
		\begin{minipage}{.5\textwidth}
			\textit{``É somente através da ajuda mútua e das concessões recíprocas que um organismo agrupando indivíduos em número grande ou pequeno pode encontrar sua harmonia plena e realizar verdadeiros progressos''.}
			\begin{flushright}
				(Jigoro Kano)
			\end{flushright}
		\end{minipage}
	\end{flushright}
\end{epigrafe}


	% ----------------------------------------------------------
% RESUMOS
% ----------------------------------------------------------

% resumo em português
\setlength{\absparsep}{18pt} % ajusta o espaçamento dos parágrafos do resumo
\begin{resumo}
 Em virtude do aumento exponencial das frotas de carros e a ampla necessidade em ter um maior controle de acesso de automóveis a ambientes fechados, é importante que sistemas de reconhecimento automático de placas veiculares possam ser utilizados de forma simples e que controlem o acesso de veículos a ambientes fechados, apenas necessitando da supervisão humana. Atualmente, torna-se viável implementar esses tipos de sistemas utilizando uma tecnologia de baixo custo. Este trabalho apresenta um método de reconhecimento automático de placas de veículos com base em reconhecimento de padrões. Utilizando um microcomputador Raspberry Pi 3 como unidade controladora do sistema, uma \textit{webcam}, um sensor ultrassônico e um motor DC de um drive de CD-ROM, para simular o portão eletrônico. Após a aquisição da imagem, o sistema segue as seguintes etapas: pré-processamento da imagem, extração da área da placa, reconhecimento dos caracteres utilizando o classificador kNN e a tomada de decisão para o acionamento do portão eletrônico. O resultado obteve um índice de reconhecimento de 92\% das placas as quais foram submetidos nos testes e um bom desempenho do sistema completo em funcionamento sendo executado em cerca de 10 segundos para a realização do processo.

 \textbf{Palavras-chaves}: Reconhecimento de Caracteres. Reconhecimento de Padrões. Visão Computacional. Classificação. Controle.
\end{resumo}

% resumo em inglês
\begin{resumo}[Abstract]
 \begin{otherlanguage*}{english}
Due to the exponential increase in car fleets and the widespread across many cities, security is needed for a greater access control of cars parking garages. It's important that automatic vehicle-plate recognition systems can be used to control vehicle access to these environments, only needing human supervision. Currently, it's feasible to implement these types of systems using a low-to-zero cost technology. This work presents a method of automatic recognition of vehicle-plates based on pattern recognition. Using a Raspberry Pi 3 microcomputer as a system controller unit, a webcam, an ultrasonic sensor and a DC motor (from a CD-ROM drive) to simulate the electronic gate, is possible to create a system capable of using the vision sensor to direct feed the source code with a vehicle-plate to check if that car with same plate can be allowed to enter in parking garage or not. After the image acquisition, the system follows the following steps: Image pre-processing, plate extraction area, character recognition using the kNN classifier and electronic gate operation through a decision making. The result shows a recognition success rate of 92\% of the plates, to which the tests were submitted and a reasonable performance of the complete system in operation, being executed in approximally 10 seconds for the accomplishment of the complete process.

   \vspace{\onelineskip}
 
   \noindent 
   \textbf{Key-words}: Characters Recognition. Pattern Recognition. Computer Vision. Classification. Control.
 \end{otherlanguage*}
\end{resumo}

% Só descomente se for necessário

% resumo em francês 
%\begin{resumo}[Résumé]
% \begin{otherlanguage*}{french}
%    Il s'agit d'un résumé en français.
% 
%   \textbf{Mots-clés}: latex. abntex. publication de textes.
% \end{otherlanguage*}
%\end{resumo}
%
%% resumo em espanhol
%\begin{resumo}[Resumen]
% \begin{otherlanguage*}{spanish}
%   Este es el resumen en español.
%  
%   \textbf{Palabras clave}: latex. abntex. publicación de textos.
% \end{otherlanguage*}
%\end{resumo}


	
	% ----------------------------------------------------------
	% inserir lista de ilustrações
	% ----------------------------------------------------------
	\pdfbookmark[0]{\listfigurename}{lof}
	\listoffigures*
	\cleardoublepage
	
	% ----------------------------------------------------------
	% inserir lista de tabelas
	% ----------------------------------------------------------
	\pdfbookmark[0]{\listtablename}{lot}
	\listoftables*
	\cleardoublepage
	
	% ----------------------------------------------------------
	% inserir lista siglas e abreviaturas
	% ----------------------------------------------------------
	% ----------------------------------------------------------
% INSERIR LISTA DE ABREVIATURAS E SIGLAS
% ----------------------------------------------------------
\begin{siglas}
  \item[ANN] 	\textit{Artificial Neural Network}.
  \item[BLE] 	\textit{Bluetooth Low Energy}.
  \item[CD] 	\textit{Compact Disk}.
  \item[DC] 	\textit{Direct Current} ou Corrente Direta.
  \item[kNN] 	\textit{k-Nearest Neighbors}.
  \item[OCR] 	\textit{Optical Character Recognition} ou Reconhecimento Óptico de Caracteres.
  \item[PDI]	Processamento Digital de Imagens.
  \item[SBC] 	\textit{Single Board Computers}.
  \item[SIFT]   \textit{Scale-Invariant Feature Transform}.
  \item[SoC] 	\textit{System-on-Chip}.
  \item[SVM] 	\textit{Support Vector Machines}.
  \item[VGA] 	\textit{Video Graphics Array}.
\end{siglas}


	
	% ----------------------------------------------------------
	% inserir lista símbolos
	%% ----------------------------------------------------------
% INSERIR LISTA DE SÍMBOLOS
% ----------------------------------------------------------
\begin{simbolos}
  \item[$ \alpha $] Alfa ou Autovalor maior de \textit{keypoint}
  \item[$ \beta $] Beta ou Autovalor menor de \textit{keypoint}
  \item[$ \sigma $] Sigma ou desvio padrão
  \item[$ \theta $] Angulo
  \item[$ \partial $] Diferencial
  \item[$ \hat{x} $] Extrema
  \item[$ \pi $] Pi ou $180^{\circ}$
\end{simbolos}


	% ----------------------------------------------------------
	
	% ----------------------------------------------------------
	% inserir o sumario
	% ----------------------------------------------------------
	\pdfbookmark[0]{\contentsname}{toc}
	\tableofcontents*
	\cleardoublepage
	
	% ----------------------------------------------------------
	% ELEMENTOS TEXTUAIS
	% ----------------------------------------------------------
	\textual
	%se não desejar o cabeçalho com informações do capitulo, descomentar as linhas seguintes.
	\pagestyle{simple}
	\aliaspagestyle{chapter}{simple}
	
	% ----------------------------------------------------------
	% Introdução (exemplo de capítulo sem numeração, mas presente no Sumário)
	% ----------------------------------------------------------
	% TODO inserir seu capítulo 1 aqui
	
	% ----------------------------------------------------------
	% ----------------------------------------------------------
% Introdução (exemplo de capítulo sem numeração, mas presente no Sumário)
% ----------------------------------------------------------
\chapter*[Introdução]{\textbf{\uppercase{Introdução}}}
\addcontentsline{toc}{chapter}{Introdução}
% ----------------------------------------------------------


O avanço tecnológico possibilita um expressivo desenvolvimento de aplicações que auxiliam o homem e as vezes até o substituem em algumas tarefas diárias. Esses avanços surgiram da necessidade do homem fazer com que as máquinas pudessem replicar habilidades humanas simplificando e automatizando tarefas complexas \cite{forsyth2012}.

Uma das habilidades humanas essenciais para o cotidiano e maximização do trabalho, é a visão. Porém, para uma máquina emular a visão humana se torna muito complexo devido a vários fatores externos como luminosidade, oclusão, distância e algoritmos que somente o cérebro humano, até o momento é capaz de replicar. Estes processos acabam dificultando a compreensão do conteúdo da imagem adquirida pela máquina.  O campo da ciência que estuda a forma como as maquinas enxergam e compreendem o mundo exterior é chamado de Visão Computacional.
 
Segundo \cite{dawson2014}, a Visão Computacional é a análise automática de imagens e vídeos por computadores, a fim de obter alguma compreensão do mundo real e é inspirada nas capacidades do sistema de visão humana.  Um dos campos da visão computacional que pode-se utilizar é o reconhecimento de padrões. Para \citeonline{theodoridis2009}, o reconhecimento de padrões é a disciplina cujo objetivo é a classificação de objetos em uma série de categorias ou classes (caracteres). Esses objetos podem ser imagens ou formas de onda de sinal ou qualquer tipo de medidas que precisam ser classificadas.

De forma geral a Visão Computacional e o reconhecimento de padrão buscam prover as máquinas conhecimento suficiente para auxiliar o homem em suas necessidades de identificar objetos automaticamente sem que, para isso, possa ter a intervenção humana.

Varias aplicações nos dias de hoje utilizam as tecnologias providas pela visão computacional e pelo reconhecimento de padrões \cite{duda2001}. Exemplos são: detecção de pessoas em certos ambientes, reconhecimento de peças e medidas por câmera, reconhecimento facial, entre outros. Um processo que chama atenção também, é o uso dessa tecnologia para o reconhecimento de veículos, por exemplo. Devido ao crescimento de frotas de veículos particulares e a facilidade em ter acesso a um \cite{dia2016,reis2014}, cresce também a demanda pelo controle e a identificação desses veículos. Visto que também a criminalidade e a falta de segurança de acesso a condomínios privados vem aumentando \cite{dia2017,prado2017,ama2017}. Para tentar diminuir esse problema e aumentar a segurança, o reconhecimento automático de placas veiculares faz-se necessário pela melhoria de segurança em determinados ambientes através do controle de acesso de quem entra e trafega em determinados locais, por exemplo.

Contudo, visto que podem existir vários problemas com relação a determinação dessas placas devido a não-uniformidade da cena (isto é, problemas de iluminação, oclusão, entre outros), deve-se treinar um sistema computacional que possa, com certa eficácia, encontrar uma placa e reconhece-la. Visto que isso é possível através de sistemas de fácil acesso e atuais \cite{luka2014}. 

%Diante desse contexto abordado um fator importante é que uma máquina dificilmente terá falhas em suas tarefas se for bem treinada para executar o que lhe foi definido. Portanto será uma máquina capaz de identificar uma placa de carro sem que haja a necessidade humana de auxilia-la?

Há de se tentar resolver estes problemas citados através do desenvolvimento de um sistema que seja capaz de identificar de forma automática os caracteres contidos nas placas de matricula veiculares. O resultado do reconhecimento dos caracteres será a base para a tomada de decisão do sistema.

Por isso, este trabalho busca construir um sistema que seja capaz de reconhecer e identificar a placa de um carro e os caracteres da placa de um veiculo e permitir que o mesmo possa, ou não, acessar um estacionamento privativo. O autor, através de várias pesquisas, buscou completar essa tarefa através da combinação de ferramentas com o \emph{framework} OpenCV, a linguagem de programação \emph{Python} e o microcomputador \emph{Raspberry Pi} executando o sistema operacional \emph{Raspbian}. Justamente para prover um sistema simples, relativamente barato e de livre acesso a possíveis replicações pelos leitores.



	
	% ----------------------------------------------------------
	% Capitulo com exemplos de comandos inseridos de arquivo externo
	% Este capitulo deve conter a contextualização, justificativa, problemática, hipoteses, objetivos e organização de seu trabalho 
	% ----------------------------------------------------------
	%% abtex2-modelo-include-comandos.tex, v-1.9.2 laurocesar
%% Copyright 2012-2014 by abnTeX2 group at http://abntex2.googlecode.com/ 
%%
%% This work may be distributed and/or modified under the
%% conditions of the LaTeX Project Public License, either version 1.3
%% of this license or (at your option) any later version.
%% The latest version of this license is in
%%   http://www.latex-project.org/lppl.txt
%% and version 1.3 or later is part of all distributions of LaTeX
%% version 2005/12/01 or later.
%%
%% This work has the LPPL maintenance status `maintained'.
%% 
%% The Current Maintainer of this work is the abnTeX2 team, led
%% by Lauro César Araujo. Further information are available on 
%% http://abntex2.googlecode.com/
%%
%% This work consists of the files abntex2-modelo-include-comandos.tex
%% and abntex2-modelo-img-marca.pdf
%%

% ---
% Este capítulo, utilizado por diferentes exemplos do abnTeX2, ilustra o uso de
% comandos do abnTeX2 e de LaTeX.
% ---
 
\chapter{\textbf{\uppercase{Contextualização}}}\label{cap_1}

Devido ao crescimento populacional e suas necessidades, as frotas de veículos também estão aumentando exponencialmente, isso reflete diretamente na necessidade de se ter um maior acesso ao controle de automóveis. Com a crescente demanda pelo aumento de segurança em locais fechados e um maior controle no acesso de seus moradores, torna-se hoje em dia imprescindível que a segurança eletrônica seja automatizada, rápida e eficaz na identificação de seus moradores. Além da observação e verificação humana, o sistema de reconhecimento de veículos através da identificação da sua placa pode aumentar a segurança. Por isso, é de suma importância que sistemas de reconhecimento automático de placas de carro sejam implementados em ambientes reservados e particulares e com um melhor controle de acesso para seus moradores, onde a intervenção humana seja minimizada ou quase não necessária nesse contexto, apenas a supervisão. 

A tecnologia atual tornou-se barata, competitiva e compacta, então torna-se fácil implementar sistemas de certa robustez em dispositivos com essas características, como é o caso do microcomputador Raspberry Pi utilizado nesse trabalho. Com sistemas e Hardwares \textit{Open Source} se popularizando, torna-se barato implementar sistemas dessa natureza aumentando o nível de segurança e controle de veículos que trafegam dentro de condomínios.

% ---
\section{\uppercase{Justificativa}}
% ---

O reconhecimento automático de placa de matricula de carros é uma tarefa importante que visa automatizar o controle de acesso em portões eletrônicos. Esse tipo de aplicação já é muito utilizado em muitos setores da sociedade, segundo \citeonline{babu2016}: ``A detecção de placas de matrícula pode ser usada em muitas aplicações, tais como controle de tráfego, controle de velocidade, identificação dos carros roubados, portões de pedágio, aplicação de segurança, etc.''.

Então, com base neste pretexto, o autor buscou um método simples e automático para tentar aumentar a segurança de sistemas que usam cancelas para acesso a determinados ambientes de estacionamento veicular. Como a aplicação utilizará basicamente uma câmera e algoritmos, o custo e a replicação do experimento são de fácil e livre acesso ao todos. Apenas algumas peças de hardware devem ser adicionadas ao projeto final em ordem de completar o processo como um produto completo para qualquer situação (dentro do escopo de acesso veicular).

% ---
\section{\uppercase{Motivação}}
% ---

Este trabalho foi motivado pela espera na identificação de um veículo e seu condutor de forma manual em duas portarias num determinado ambiente reservado, onde se faz a necessidade da criação de aplicações automatizadas para o controle de acesso a lugares reservados de modo rápido, preciso e quase sem a necessidade da intervenção humana através de uma forma eficiente de se obter informação ao acesso de veículos. Visto que, é comum nos dias de hoje haverem acessos indevidos a condomínios e até mesmo a residências urbanas por estranhos sem autorização e ladrões. O que intriga o autor, é que tecnologias com esta citada no trabalho, já são bem comuns em cidades do Brasil a anos e na sua cidade, Manaus-AM, é raro ver esse tipo de adicional a segurança, mesmo em condomínios e prédios relativamente novos.

% ---
\section{\uppercase{Problemática}}
% ---

A demora na identificação de um veículo e seu condutor feito de forma manual e acessos indevidos em ambientes fechados e um sistema de controle eficiente, são problemas que o autor tenta buscar a solução através do um desenvolvimento possível de um sistema de reconhecimento automático de placas veiculares, que é uma tecnologia bastante usada hoje em dia pela computação, mas pouco explorada por acadêmicos de engenharia elétrica. Por ser, geralmente, de código aberto e gratuito, os algoritmos que são utilizados na Visão Computacional podem solucionar o problema com eficácia, na visão do autor.  

% ---
\section{\uppercase{Hipóteses}}
% ---

\begin{itemize}
	\item A utilização de uma \textit{Webcam} de baixo custo poderá afetar na resolução da imagem adquirida. Contudo, é possível adicionar algumas etapas de processamento digital de imagens com uso de filtros de correção de cor e de iluminação para maximizar a qualidade da imagem;
	\item Poderá haver erros na identificação dos caracteres nas placas dos carros, devido ao fato que nenhum método é perfeito. O algoritmo de treinamento de identificação dos caracteres pode não reconhecer corretamente ou haver alguma distorção na imagem que impossibilite o reconhecimento de algum caractere. Ainda lembrando que, dependendo da fonte utilizada na placa, o treinamento de caracteres pode perder muito a taxa de acerto. A fonte deve ser levada em conta aqui;
	\item Pode ser possível que, mesmo com todas as soluções apontadas, a câmera acabe não reconhecendo a placa devido alguns problemas na cena. Por isso, a câmera ficará colocada em posição bem favorável para captura correta da imagem da placa. Talvez, seja necessário ajustar o ambiente para melhorar a captura, como iluminação artificial e posicionamento da câmera para evitar oclusões.
\end{itemize}

% ---
\section{\uppercase{Objetivos}}
% ---
Serão apresentados os seguintes objetivos deste trabalho a partir das análises realizadas inicialmente para essas abordagens.

\subsection{\textbf{Objetivo Geral}}

Desenvolver um sistema de reconhecimento automático da placas de matricula veicular para aumentar a segurança e o controle de acesso de veículos em ambientes reservados. 

\subsection{\textbf{Objetivos Específicos}}

\begin{itemize}
	\item [\textbf{a.}] Aumentar a performance do algoritmo de detecção de caracteres em ordem de diminuir os erros;
	\item [\textbf{b.}] Desenvolver um protótipo de um portão eletrônico para simular o funcionamento do sistema e construir o algoritmo de acionamento do mesmo;
	\item [\textbf{c.}] Realizar a comparação de custo de construção do protótipo do sistema com outros sistemas prontos no mercado para verificar se o valor de acesso se mantem baixo e de fácil acesso.
\end{itemize}

\section{\uppercase{Organização do Trabalho}}

Este trabalho está estruturado em ordem sequencial com uma breve descrição dos conteúdos dos capítulos:

\begin{itemize}
	\item \textbf{Fundamentação Teórica:} é o capitulo 2 deste trabalho. Apresenta conceitos e definições de assuntos relacionados a visão computacional e suas etapas. São estudadas as teorias relacionados ao processo de reconhecimento da imagem, das ferramentas (aplicativos e as bibliotecas utilizadas do OpenCV) e do hardware empregado neste trabalho;
	\item \textbf{Trabalhos Correlatos:} é o capitulo 3 deste trabalho. Mostra vários trabalhos que foram abordados na área de visão computacional que serviram de embasamento para a busca de uma solução no reconhecimento automático de placas veiculares utilizando reconhecimento de padrões;
	\item \textbf{Método Proposto:} é o capitulo 4 deste trabalho. Aqui é abordado a forma de como chegar a solução do objetivo geral proposto, apresentando os métodos para alcançar a resolução desejada; 
	\item \textbf{Resultados:} é o capitulo 5 deste trabalho. Apresenta os resultados obtidos após a conclusão do objetivo geral e do específicos que foram concluídos, falando também das suas possíveis limitações;
	\item \textbf{Conclusão e projetos futuros:} finaliza o trabalho trazendo abordagens sobre o sucesso dos resultados obtidos e possíveis falhas nos objetivos específicos que não foram alcançados, sugerindo uma melhoria no método proposto e trabalhos futuros que venham agregar alguma possível melhoria ao projeto.
\end{itemize}

	
	% ----------------------------------------------------------
	% Capitulo com exemplos de comandos inseridos de arquivo externo
	% Este capitulo deve conter a fundamentação téorica de seu trabalho 
	% ----------------------------------------------------------
	%\chapter{\textbf{\uppercase{Fundamentação Teórica}}}\label{cap_2}

Neste capitulo será contextualizado as informações teóricas para a compreensão deste projeto. O capitulo deve prover ao leitor a base teórica para compreender os conceitos que serão melhor explicados ao longo deste capítulo para poder familiarizar com os futuros tópicos vistos em outros capítulos.

\section{\uppercase{Visão Computacional}}

A visão computacional busca igualar ou simular a visão humana na interpretação das cenas do mundo real. Segundo~\citeonline{dawson2014}, a visão computacional é a análise automática de imagens e vídeos com o intuito de obter alguma informação ou dado do ambiente que cerca o sensor de visão. Assim, a visão computacional prove a capacidade de extrair informações de uma imagem que possa servir de base direta ou indireta em tomadas de decisões futuras, como por exemplo orientação de um robô com sensores visuais.

Segundo~\citeonline{forsyth2012}, a visão computacional tem uma grande variedade de aplicações, tanto antigas (por exemplo, robô móvel de navegação, inspeção industrial e inteligência militar) como novas (por exemplo, interação com computadores humanos, recuperação de imagens em bibliotecas digitais, análise de imagens médicas e a renderização realista de cenas em computação gráfica).

Um exemplo de aplicação de visão computacional seria o reconhecimento de caracteres em uma placa de um carro conforme demonstra a~\autoref{fig:id_placa} mostra e que através de uma série de processos na imagem chega-se a um resultado a ser avaliado (tomada de decisão). 

\begin{figure}[htb]
	\centering
	\caption{{\footnotesize Identificação dos Caracteres da Placa Veicular.}}   % nome da figura
	\label{fig:id_placa}
	\includegraphics[width=.7\textwidth]{1.pdf}
	
	%\legend{Fonte: \citeonline{Vestibulando_2016}.}   % outra opção de legenda da foto
	{\footnotesize Fonte: \citeonline{araujo2015}.}
\end{figure}

\section{\uppercase{Etapas do Processo}}

Para se chegar a um resultado satisfatório no reconhecimento de imagem, se faz necessário que a imagem passe por processos importantes iniciados pela aquisição da imagens (ou vídeos), pré-processamento, segmentação, seleção e extração de características e classificação, até chegar ao resultado. A descrição dos processos é mostrada na \autoref{fig:etapas} conforme a sequência de processamento. Estes passos serão melhor explanados ao longo deste capitulo.

\begin{figure}[htb]
	\centering
	\caption{{\footnotesize Etapas do processo para o reconhecimento de imagens.}}   % nome da figura
	\label{fig:etapas}
	\includegraphics[width=.7\textwidth]{2.pdf}
	
	%\legend{Fonte: \citeonline{Vestibulando_2016}.}   % outra opção de legenda da foto
	{\footnotesize Fonte: Próprio Autor, 2016 (Adaptado de~\cite{juliana2016}).}
\end{figure}

\subsection{\textbf{Pré-processamento}}

A imagem ou vídeo adquirido em meio as condições do processo de aquisição pode conter imperfeições. No intuito de utilizar essa imagem nas demais fases do processo. De acordo com~\cite{pedrini2017} que afirma:

\begin{citacao}
	
	A etapa de pré-processamento visa melhorar a qualidade da imagem por meio da aplicação de técnicas para atenuação de ruído, correção de contraste ou brilho e suavização de determinadas propriedades da imagem.
	
\end{citacao}
	 

\subsection{\textbf{Segmentação}}

Para~\citeonline{forsyth2012}, a ideia central de segmentação é coletar \emph{pixels} ou elementos de padrão em representações sumárias que enfatizam propriedades importantes, interessantes e distintivas. Assim, conforme~\citeonline{sobral2003} a segmentação tem por objetivo em regiões ou objetos, frequentemente ocasionando não uma imagem como resultado, mas um conjunto de regiões ou objetos.

Segundo~\citeonline{agarwal2016}, o processo de segmentação pode ser feito em duas abordagens distintas: por descontinuidade, que seria a segmentação de bordas e contornos ou; por similaridade que seria, binarização, crescimento de regiões, divisão e junção de regiões. A~\autoref{fig:proc_gradiente} ilustra um exemplo do processo de segmentação por descontinuidade.

\begin{figure}[htb]
	\centering
	\caption[\footnotesize Processo de Segmentação por Descontinuidade.]{\footnotesize Processo de Segmentação por Descontinuidade. a) Imagem de entrada; b) $G_y$ Componente do gradiente vertical; c) $G_x$ Componente do Gradiente horizontal; d) Resultado dos contornos ligados entre si.}   % nome da figura
	\label{fig:proc_gradiente}
	\begin{subfigure}{.4\textwidth}
		\centering
		\includegraphics[width=.7\linewidth]{3a.pdf}
		\caption{ }
	\end{subfigure}
	\begin{subfigure}{.4\textwidth}
		\centering
		\includegraphics[width=.7\linewidth]{3b.pdf}
		\caption{ }
	\end{subfigure}
	\\
	\begin{subfigure}{.4\textwidth}
		\centering
		\includegraphics[width=.7\linewidth]{3c.pdf}
		\caption{ }
	\end{subfigure}
	\begin{subfigure}{.4\textwidth}
		\centering
		\includegraphics[width=.7\linewidth]{3d.pdf}
		\caption{ }
	\end{subfigure}
	\\
	%\legend{Fonte: \citeonline{Vestibulando_2016}.}   % outra opção de legenda da foto
	{\footnotesize Fonte:~\citeonline{sobral2003}.}
\end{figure}

\subsection{\textbf{Seleção e Extração de Características}}

Seleção de características também é chamada seleção de variável ou seleção de atributo. É a seleção automática de atributos e seus dados que são mais relevantes para o problema de modelagem, é diferente da redução de dimensionalidade \cite{brownlee2014}. Os dois métodos procuram reduzir o número de atributos no conjunto de dados, mas um método de redução de dimensionalidade o faz criando novas combinações de atributos, onde como métodos de seleção de recursos incluem e excluem atributos presentes nos dados sem alterá-los \cite{brownlee2014}.

Existem dois métodos principais para reduzir a dimensionalidade: seleção de características e extração de características. Na seleção de características, tem-se o interesse de encontrar $k$ características das dimensões $d$ que dão mais informações e descartamos as outras dimensões $(d - k)$ \cite{alpaydin2014}.

Segundo \citeonline{alpaydin2014}, esses métodos podem ser supervisionados ou não supervisionados dependendo se eles usam ou não as informações de saída. 

\subsection{\textbf{Classificação ou Reconhecimento de Padrão}}

Reconhecimento de padrão de acordo com \citeonline{agarwal2016} é o processo de classificação de entrada de dados dentro de uma classe sendo essa classificação baseada em dois métodos aprendizagem.

Segundo \cite{bianchi2006}, há duas formas de conhecimento de padrão: classificação supervisionada; onde o padrão de entrada tem sua identificação Projetada para uma classe pré-definida ou classificação não supervisionada; onde o padrão é estabelecido por uma fronteira de classe desconhecida. 

Para \citeonline{duda2001}, uma das fases do reconhecimento padrão é a classificação de objetos, caracterizando e agrupando por categorias o objeto de interesse ao modelo, podendo ter um treinamento anterior ou não para a classificação.

Uma dificuldade no reconhecimento de padrão consiste na tarefa de categorização, onde as classes são definidas pelo projetista do sistema (classificação supervisionada) ou são ``aprendidas'' de acordo com a similaridade dos padrões (classificação não supervisionada) \cite{bianchi2006}.

Com o crescente avanço tecnológico, cresceu também o interesse em classificação de padrões, por permitir que aplicações computacionais possam auxiliar em diversas áreas do conhecimento. A \autoref{tab:padrao} mostra exemplos de aplicações com classes de padrões.

 \begin{table}[htb]
	\ABNTEXfontereduzida
	\caption[\footnotesize Exemplos de aplicações para o reconhecimento de padrões.]{\footnotesize Exemplos de aplicações para o reconhecimento de padrões.}
	\label{tab:padrao}
	\begin{tabular}{p{3cm}|p{3cm}|p{4cm}|p{4cm}}
		\hline
		\textbf{Área} & \textbf{Utilização} & \textbf{Entradas de Dados} & \textbf{Classes Padrão} \\
		\hline \hline
		\textbf{Bioinformática} & Análise de sequência	& DNA/Sequência de Proteína	& Tipos conhecidos de genes/padrões \\
		\hline
		\textbf{Mineração de dados} & Busca por padrões Significantes & Pontos em um espaço multidimensional & Compactar e separar grupos \\
		\hline
		\textbf{Classificação de documentos} & Busca na Internet & Documento texto & Categorias semânticas (negócios, entre outros) \\
		\hline
		\textbf{Análise de documento de imagem} & Máquina de leitura para cegos & Imagem de Documentos & Caracteres alfanuméricos, palavras \\
		\hline
		\textbf{Automação industrial} & Inspeção de placas de circuito impresso & Intensidade ou alcance de imagem & Natureza do produto (defeituosa ou não) \\ 
		\hline
		\textbf{Recuperação de base de dados multimídia} & Busca pela Internet & \textit{Video Clip} & Gêneros de vídeo (p.e. ação, diálogo, entre outros.) \\
		\hline
		\textbf{Reconhecimento biométrico} & Identificação pessoal & Face, íris, impressão digital & Usuários autorizados para controle de acesso \\
		\hline
		\textbf{Sensoriamento remoto} & Prognóstico da produção de colheita & Imagem multiespectral & Categorias de aproveitamento de terra, desenvolvimento de padrões de colheita \\
		\hline
		\textbf{Reconhecimento de voz} & Inquérito por telefone sem assistência de operador & Voz em forma de onda & Palavras faladas \\
		\hline \hline
	\end{tabular}
	\legend{\footnotesize Fonte: Próprio autor, 2017 (adaptado de \cite{bianchi2006}).}
\end{table} 

\section{\uppercase{Classificador}}

\subsection{\textbf{k-Nearest Neighbors (kNN)}}

Segundo~\citeonline{rosebrock2016}, kNN é de longe o mais simples algoritmo de classificação de imagem e aprendizado de máquina. A operação simplesmente compara um objeto a ser classificado com todos os objetos em uma série de treinamentos com rótulos de classes já conhecidas e tenta indicar o voto para qual classe atribuir o objeto \cite{solem2012}.

Para~\citeonline{luiza2005}, kNN é um classificador que tem como base de aprendizado à similaridade. O Conjunto de treinamento é formado por vetores $n$-dimensionais e cada elemento deste conjunto representa um ponto no espaço $n$-dimensional. Afim de obter a classe de um elemento que seja desvinculado ao conjunto de treinamento, o algoritmo busca por $K$ elementos do conjunto de treinamento tenham a menor distância. Ainda sobre as palavras diretas de \cite{luiza2005}: ``Estes $K$ elementos são chamados de $K$-vizinhos mais próximos (kNN). Verifica-se quais são as classes desses $K$ vizinhos e a classe mais frequente será atribuída à classe do elemento desconhecido.''

A seguir, apresenta-se as métricas comumente utilizadas no cálculo da distância entre dois pontos. O método utilizado com mais frequência é a distância Euclidiana \autoref{eq:euclid}. Outras métricas como as de Manhattan e de Minkowski, são apontadas nas \autoref{eq:manh} e \autoref{eq:mink}, respectivamente. O ultimo é uma generalização das duas distâncias anteriores.

Seja uma imagem $I$ de tamanho $X \times Y$ e $X=(x_1,x_2,x_3,\ldots,x_n)$ e $Y=(y_1,y_2,y_3,\ldots,y_n)$, dois conjuntos de pontos no plano $\Re^k$, onde $k$ é 2 se a imagem for bidimensional.

\begin{equation} \label{eq:euclid}
D(x,y) = \sqrt{(x_1-y_1)^2 + (x_2-y_2)^2 + \ldots + (x_n - y_n)^2}
\end{equation}

\begin{equation} \label{eq:manh}
D(x,y) = |x_1-y_1|+|x_2-y_2|+\ldots+|x_n-y_n|
\end{equation}

\begin{equation} \label{eq:mink}
D(x,y) = \sqrt[q]{|x_1-y_1|^q + |x_2-y_2|^q + \ldots + |x_n-y_n|^q} \rightarrow q \in \aleph
\end{equation}

Quando $q = 1$, a distância de Minkowski representa a distância de Manhattan e quando $q = 2$, a distância Euclidiana.

A distância Euclidiana ponderada também pode ser representada conforme~\citeonline{luiza2005} cita, se cada variável possuir um peso relativo à sua importância:

\begin{citacao}
	
	Pesos também podem ser aplicados às distâncias Manhattan e Minkowski. O kNN é um classificador que possui apenas um parâmetro livre (o número de K-vizinhos) que é controlado pelo usuário com o objetivo de obter uma melhor classificação. Este processo de classificação pode ser computacionalmente exaustivo se considerado um conjunto com muitos dados. Para determinadas aplicações, no entanto, o processo é bem aceitável.	

\end{citacao}

A \autoref{fig:knn_ex} mostra um exemplo de classificação pelo algoritmo kNN com as seguintes descrições:

\begin{itemize}
	\item Dois atributos;
	\item Três classes;
	\item Dois pontos desconhecidos (1 e 2); e
	\item Deseja-se classificar estes dois pontos através dos 7 vizinhos mais próximos.
\end{itemize}

\begin{figure}[htb]
	\centering
	\caption{{\footnotesize Classificação pelo método kNN.}}   % nome da figura
	\label{fig:knn_ex}
	\includegraphics[width=.7\textwidth]{4.pdf}
	
	%\legend{Fonte: \citeonline{Vestibulando_2016}.}   % outra opção de legenda da foto
	{\footnotesize Fonte:~\citeonline{luiza2005}.}
\end{figure}

\section{\uppercase{Framework OpenCV}}

OpenCV é um conjunto de bibliotecas \emph{Open-Source} de visão computacional e de aprendizagem de máquina \cite{pavalenko2017} que ajuda a cientistas e pesquisadores na área de processamento digital de imagens (PDI) a resolver problemas usando uma linguagem de programação livre e que possui uma comunidade enorme em termos de exemplos e aplicações educacionais e até mesmo, comerciais.

Segundo \citeonline{pavalenko2017}, o \emph{framework} possui hoje em dia mais de 2500 algoritmos clássicos e de estado da arte em visão computacional, PDI e aprendizado de máquina. Inicialmente o OpenCV começou como uma biblioteca desenvolvida para o Fortran pela Intel em 1999. Porém, esse sistema foi abandonado em meados de 2000 e, posteriormente, tido o suporte de colaboradores da \emph{Willow Garage}, que mais tarde se tornaria na OpenCV Foundation. O objetivo do software é diminuir o \emph{Gap} entre a visão computacional e programadores ao redor do mundo, principalmente quando se trata de aplicações comerciais e fechadas providas pela MathWorks e outras.

Ainda de acordo com \citeonline{pavalenko2017}, o software OpenCV foi escrito em C++, mas pode ter suas aplicações desenvolvidas nas linguagens C, Python, Java e MATLAB. Além de poder ser instalado em plataformas como Windows, Linux, Android e Mac OS. Hoje a mesma se encontra na versão 3.2 encontrada tanto no \href{http://opencv.org/releases.html}{site oficial} quanto no \href{https://github.com/opencv/opencv}{GitHub}. 

\section{\uppercase{System-on-Chip: Raspberry Pi}}

O Raspberry Pi é um microcomputador da família dos \textit{Single Board Computers} (SBC) do tamanho de um cartão de crédito que foi desenvolvido no Reino Unido pela Raspberry Pi Foundation \cite{pajankar2017}.

Ainda de acordo com \citeonline{pajankar2017}, objetivo por trás da criação da criação do Raspberry Pi foi de promover o ensino básico de ciência da computação em escolas de países em desenvolvimento fornecendo uma plataforma de computação de baixo custo. Por isso, a plataforma foi escolhida, visto que o ultimo modelo lançado (em fevereiro de 2016) suporta tanto sistemas operacionais Windows e Linux. Além de ser o sistema de \emph{System-on-Chip} (SoC) de alta performance mais utilizado no mundo.

Na \autoref{tab:rasp_specs}, mostra as especificações do modelo do Raspberry Pi mais recente (que no caso é o modelo 3B) e que será utilizado neste trabalho a fim desenvolver o projeto proposto.

\begin{table}[htb]
	\ABNTEXfontereduzida
	\caption[\footnotesize Especificações Técnicas do Raspberry Pi 3 Model B.]{\footnotesize Especificações técnicas do Raspberry Pi 3 Model B.}
	\label{tab:rasp_specs}
	\centering
	\begin{tabular}{c|p{11cm}}
		\hline
		\textbf{Item} & \textbf{Descrição} \\
		\hline \hline
		\textbf{Processador} & 1.2 GHz Quad-Core ARM Cortex-A53 de 64 bits\\
		\hline
		\textbf{Chipset} & Broadcom BCM2387 \\
		\hline
		\textbf{Wifi} & 802.11 b/g/n Wireless LAN \\
		\hline
		\textbf{LAN} & 10/100 BaseT Ethernet  \\
		\hline
		\textbf{Bluetooth} & Bluetooth 4.1 (Bluetooth Clássico e BLE) \\ 
		\hline
		\textbf{RAM} & 1GB RAM DDR3 \\
		\hline
		\textbf{Portas} & HDMI, P2 para áudio estéreo, Vídeo Composto, Conexão CSI para câmera, Conexão DSI para display de toque e 4 USB \\
		\hline
		\textbf{Fonte} & Conexão de fonte bivolt de 5VDC microUSB \\
		\hline
		\textbf{Cartão de Memoria} & Suporte a SD Card de até 256GB \\
		\hline \hline
	\end{tabular}
	\legend{\footnotesize Fonte: Próprio autor, 2017 (adaptado de \cite{pajankar2017}).}
\end{table} 

A \autoref{fig:rasp_specs} demonstra com mais detalhes a placa do Raspberry Pi 3 com a descrição de seus elementos.

\begin{figure}[htb]
	\centering
	\caption{{\footnotesize Detalhes do Raspberry Pi 3 Model B.}}   % nome da figura
	\label{fig:rasp_specs}
	\includegraphics[width=.8\textwidth]{5.pdf}
	
	%\legend{Fonte: \citeonline{Vestibulando_2016}.}   % outra opção de legenda da foto
	{\footnotesize Fonte:~\citeonline{element2016}.}
\end{figure}

\section{\uppercase{Considerações Finais}}

O escopo deste capitulo foi apresentar de forma sucinta sobre visão computacional, as etapas do processo de visão computacional até a tomada do resultado, discorrer um pouco sobre a biblioteca OpenCV a ser utilizada nesse trabalho, sobre o microcomputador Raspberry Pi que terá o seu papel como o processador central das tarefas que virão na proposta objetiva desse trabalho.

Este capitulo também servirá de base compreensiva para capítulos posteriores que apresentarão nomenclaturas relacionadas proposição da melhor maneira de resolver os problemas que possam surgir ao longo desse trabalho.

	% ----------------------------------------------------------
	
	% ----------------------------------------------------------
	% Capitulo com exemplos de comandos inseridos de arquivo externo
	% Este capitulo deve conter os trabalhos correlatos ao seu trabalho 
	% ----------------------------------------------------------
	%\chapter{\textbf{\uppercase{Trabalhos Correlatos}}}\label{cap_3}

Neste capítulo, será apresentado trabalhos correlacionados com esse trabalho, onde diversas abordagens de autores que já publicaram suas experiencias no tópico abordado, servirão de embasamento metodológico para a solução das etapas aqui empregadas no processo do reconhecimento automático das placas veiculares. Os trabalhos selecionados apresentam diversas formas de se chegar a uma possível solução com uso da visão computacional como ferramenta principal.

A pratica da revisão sistemática contribuiu bastante para que se possa adquirir um certo grau de conhecimento através de um acervo de artigos científicos pelo qual classificou-se as abordagens mais pertinentes, a fim de prover uma base sólida sobre o experimento e estudos que já foram feitos por autores previamente antes deste trabalho. Além de prover o que ainda pode ser melhorado e contribuir para comunidade cientifica como um todo.

\section{\uppercase{Categorização dos Trabalhos Analisados}}

\subsection{\textbf{Enumeração dos Trabalhos Analisados}}

Nos artigos analisados foram identificadas as quatro características principais que foram classificadas como categorias e várias subdivisões por métodos adotados pelos autores.

Os artigos, \emph{a priori}, foram enumerados para uma melhor organização do trabalho conforme foram lidos e analisados. Na \autoref{tab:correla} mostra como foi organizado cada linha através dos autores, tema e ano de publicação.

\begin{table}[htb]
	\ABNTEXfontereduzida
	\caption[\footnotesize Enumeração dos trabalhos analisados.]{\footnotesize Enumeração dos trabalhos analisados.}
	\label{tab:correla}
	\begin{tabular}{c|c|p{6cm}|c}
		\hline
		\textbf{\#} & \textbf{Autor(es)} & \textbf{Tema} & \textbf{Ano}\\
		\hline \hline
		1 & \cite{agarwal2016} & An Efficient Algorithm for Automatic Car Plate Detection \& Recognition & 2016 \\
		\hline
		2 & \cite{jia2016} & Design Flow of Vehicle License Plate Reader Based on RGB Color Extractor & 2016 \\
		\hline
		3 & \cite{babu2016} & Vehicle Number Plate Detection and Recognition using Bounding Box Method & 2016 \\
		\hline
		4 & \cite{george2016} & VNPR system using Artificial Neural Network & 2016 \\
		\hline
		5 & \cite{khan2016} & Car Number Plate Recognition (CNPR) System Using Multiple Template Matching & 2016 \\
		\hline
		6 & \cite{ikeizumie2014} & An Effective Sequence of Operations for License Plates Recognition & 2014 \\
		\hline
		7 & \cite{trentini2010} & Reconhecimento Automático de Placas de Veículos & 2010 \\
		\hline
		8 & \cite{islam2015} & Automatic Vehicle Number Plate Recognition Using	Structured Elements & 2015 \\
		\hline
		9 & \cite{khan2016comparison} & Comparison of Various Edge Detection Filters for ANPR & 2016 \\
		\hline
		10 & \cite{ha2016} & License Plate Automatic Recognition Based on Edge Detection & 2016 \\
		\hline
		11 & \cite{saleem2016} & Automatic License Plate Recognition Using Extracted Features & 2016 \\
		\hline
		12 & \cite{sen2014} & Advanced License Plate Recognition System for	Car Parking & 2014 \\
		\hline
		13 & \cite{singh2015} & ANPR Indian system using Surveillance Cameras & 2015 \\
		\hline \hline
	\end{tabular}
	\legend{\footnotesize Fonte: Próprio autor, 2016.}
\end{table} 

\subsection{\textbf{Quantização dos Métodos Aplicados}}

Nesta seção, a \autoref{tab:quant} apresenta a frequência das ocorrências dos métodos utilizados ao alongo do processo de reconhecimento de padrão, que no casso especifico desse trabalho será reconhecer os caracteres contidos nas placas veiculares. As frequências dos métodos adotados servirão de parâmetros para que se possa fazer a escolha da metodologia que mais se assemelha ao objetivo desse trabalho.

Para que o leitor possa entender melhor a \autoref{tab:quant}, o autor categorizou os processos conforme os tópicos das principais fases da metodologia adotada pelos autores de cada trabalho analisado.

Logo as categorias ou etapas foram nomeadas da seguinte forma: 

\begin{itemize}
	\item PP = Pré-Processamento;
	\item DLP = Detecção e localização da Placa;
	\item SECPC = Segmentação e Extração de Características da Placa e dos Caracteres; e
	\item RC = Reconhecimento de Caracteres.
\end{itemize}

Os algoritmos e técnicas empregados em cada etapa foram relacionados as suas respectivas categorias dentro de cada trabalho.

\begin{table}[htb]
	\ABNTEXfontereduzida
	\caption[\footnotesize Quantização dos métodos adotados.]{\footnotesize Quantização dos métodos adotados em cada artigo analisado.}
	\label{tab:quant}
	\begin{tabular}{p{1.7cm}|p{2.7cm}|p{2cm}|p{.1cm}|p{.1cm}|p{.1cm}|p{.1cm}|p{.1cm}|p{.1cm}|p{.1cm}|p{.1cm}|p{.1cm}|p{.1cm}|p{.1cm}|p{.1cm}|p{.1cm}|p{.7cm}|p{.4cm}}
		\hline
		\multirow{2}{*}{\textbf{Categorias}} & \multicolumn{2}{c}{\multirow{2}{*}{\textbf{Subcategorias}}} & \multicolumn{13}{|c|}{\textbf{Artigos}} & \multirow{2}{*}{\textbf{Total}} & \multirow{2}{*}{\textbf{\%}} \\
		\cline{4-16}
		& & & 1 & 2 & 3 & 4 & 5 & 6 & 7 & 8 & 9 & 10 & 11 & 12 & 13 & & \\
		\hline \hline
		\multirow{4}{*}{PP} & \multicolumn{2}{p{4.7cm}|}{Conversão de RGB para Escala de Cinza} & X & X & X & X & X & X & X & X & X & X & X & X & X & 13 & 100\% \\
		\cline{2-18}
		& Melhoria de Contraste & \textit{Histogram Equalization} &  & X &  &  &  &  &  &  &  & X &  &  &  & 2 & 15\% \\
		\cline{2-18}
		& \multirow{2}{*}{Redução de Ruído} & \textit{Median Filter} & & X & X &  &  &  & X &  & X & X &  &  &  & 5 & 38\% \\
		\cline{3-18}
		& & \textit{Gaussian Filter} & &  &  &  &  & X &  &  &  &  &  &  & X & 2 & 15\% \\
		\cline{2-18}
		& Rotação da Imagem & \textit{Hough Line Transformation} & & X &  &  &  &  &  &  &  &  &  &  &  & 1 & 8\% \\
		\hline
		\multirow{2}{*}{DLP} & \multirow{2}{*}{\textit{Edge Detection}} & \textit{Canny Edge Detection} & X &  &  &  &  &  & X & X &  & X &  & X &  & 5 & 38\% \\
		& & \textit{Sobel Edge Filter} &  &  & X &  & X &  &  &  &  &  & X &  & X & 4 & 31\% \\
		\cline{2-18}
		& \textit{Extract Plate} & \textit{Pixel Statistics Method} & & X &  &  &  &  &  &  &  &  &  &  &  & 1 & 8\% \\
		\hline
		\multirow{6}{*}{SECPC} & \multicolumn{2}{p{4.7cm}|}{Operações Morfológicas de Erosão e Dilatação} & X & X &  &  &  & X & X & X &  &  &  &  & X & 6 & 46\% \\
		\cline{2-18}
		& \multicolumn{2}{p{4.7cm}|}{\textit{Robert Operator}} &  & X &  &  &  &  &  &  &  &  &  &  &  & 1 & 8\% \\
		\cline{2-18}
		& \multicolumn{2}{p{4.7cm}|}{\textit{Global Threshold Method}} &  & X &  &  &  &  &  &  &  &  &  &  &  & 1 & 8\% \\
		\cline{2-18}
		& \multicolumn{2}{p{4.7cm}|}{\textit{Bounding Box Method}} & X &  & X &  &  &  &  &  &  &  &  &  &  & 2 & 15\% \\
		\cline{2-18}
		& \multicolumn{2}{p{4.7cm}|}{\textit{Projection Method}} &  &  &  & X &  & X &  &  &  &  &  &  &  & 2 & 15\% \\
		\cline{2-18}
		& \multicolumn{2}{p{4.7cm}|}{\textit{Otsu Method}} &  &  &  & X &  & X &  &  &  &  &  &  & X & 3 & 23\% \\
		\hline
		\multirow{6}{*}{RC} & \multicolumn{2}{p{4.7cm}|}{Template Matching} & X &  & X &  & X &  &  &  & X & X & X &  &  & 6 & 46\% \\
		\cline{2-18}
		& \multicolumn{2}{p{4.7cm}|}{\textit{Tesseract Tool}} &  & X &  &  &  &  &  &  &  &  &  &  &  & 1 & 8\% \\
		\cline{2-18}
		& \multicolumn{2}{p{4.7cm}|}{ANN} &  &  &  & X &  &  &  &  &  &  &  &  &  & 1 & 8\% \\
		\cline{2-18}
		& \multicolumn{2}{p{4.7cm}|}{\textit{Random Trees or Random Forest}} &  &  &  &  &  &  & X &  &  &  &  &  &  & 1 & 8\% \\
		\cline{2-18}
		& \multicolumn{2}{p{4.7cm}|}{SVM} &  &  &  &  &  &  &  &  &  &  &  &  & X & 1 & 8\% \\
		\cline{2-18}
		& \multicolumn{2}{p{4.7cm}|}{SIFT} &  &  &  &  &  & X &  &  &  &  &  &  &  & 1 & 8\% \\
		\hline \hline
	\end{tabular}
	\legend{\footnotesize Fonte: Próprio autor, 2016.}
\end{table} 

Observou-se estatisticamente através da \autoref{tab:quant} quais os métodos mais utilizados em cada parte do processo de reconhecimento automático de placas veiculares. No Pré-processamento, a conversão da imagem no sistema RGB para escala de tons de cinza foi de 100\%, o uso do \textit{Median Filter} para remoção de ruídos foi de 38\% e na melhoria de contraste o \textit{Histogram Equalization} teve 15\% de utilização nos artigos revisados.

No processo de detecção e localização da placa o método mais usado foi o \textit{Canny Edge detection} com 38\% de uso, seguido pelo filtro Sobel com 31\%. Na parte de segmentação de placa ou caracteres as operações morfológicas tiveram uma frequência de 46\% seguido pelo método Otsu de 23\%. Já na fase de Reconhecimento de caracteres o algoritmo de \textit{Template Matching} teve incidência de uso na faixa de 46\%. 

Todo esse levantamento teve como base os treze artigos lidos e revisados, a fim de encontrar um algoritmo de sistema de reconhecimento de caracteres em placas veiculares que possa contribuir para a realização deste trabalho.

\section{\uppercase{Descrição das metodologias aplicadas nos trabalhos correlacionados}}

\subsection{\textbf{Pré-Processamento}}

A maioria dos trabalhos analisados iniciam suas etapas de pré-processamento das imagens através da aquisição da imagem por algum dispositivo ótico, geralmente por uma câmera de vigilância ou um \textit{webcam} passando em seguida para a próxima etapa. Dentro do processo de pré-processamento que é a fase de melhoria da imagem adquirida no sistema de cores RGB.

Em seu trabalho, \citeonline{agarwal2016}, para aumentar a performance de processamento do sistema, a fase inicial do pré-processamento teve duas etapas. Foi realizada a conversão da imagem capturada no sistema de cores RGB para tons de cinza. Os ruídos na imagem original são inevitáveis, porém para eliminar esses ruídos utilizou-se o filtro de média. A imagem da placa apresentava ruídos característicos de sal e pimenta. Para remover esse ruído, o filtro de média é mais recomendado. A \autoref{fig:pre_proc} mostra essa fase de pré-processamento na imagem da placa.

\begin{figure}[htb]
	\centering
	\caption[\footnotesize Etapas do Pré-processamento de uma Imagem da Placa Veicular.]{\footnotesize Etapas do pré-processamento de uma imagem da placa veicular. a) Imagem Original; b) Imagem original em tons de Cinza; c) Imagem filtrada com o Filtro de Media.}   % nome da figura
	\label{fig:pre_proc}
	\begin{subfigure}{.4\textwidth}
		\centering
		\includegraphics[width=.7\linewidth]{6a.pdf}
		\caption{ }
	\end{subfigure}
	\begin{subfigure}{.4\textwidth}
		\centering
		\includegraphics[width=.7\linewidth]{6b.pdf}
		\caption{ }
	\end{subfigure}
	\\
	\begin{subfigure}{.4\textwidth}
		\centering
		\includegraphics[width=.7\linewidth]{6c.pdf}
		\caption{ }
	\end{subfigure}
	\\
	%\legend{Fonte: \citeonline{Vestibulando_2016}.}   % outra opção de legenda da foto
	{\footnotesize Fonte:~\citeonline{agarwal2016}.}
\end{figure}

O trabalho de \citeonline{jia2016} também aplica o filtro de média para a remoção efetiva de ruídos, visando aumentar e melhorar a visibilidade da imagem após a conversão da imagem em tons de cinza. Para \citeonline{babu2016} expõe que pode usar o filtro de média para eliminar outros ruídos além do de sal e pimenta, mas não só eliminar o ruído, mas concentra-se também na alta frequência.

Em seu trabalho, \citeonline{singh2015} utilizou um filtro diferente para a remoção de ruído e suavização da imagem, o filtro gaussiano para borrar a imagem e suavizar ruídos de alta frequência. Esse filtro é muito utilizado na literatura para os mesmos fins que o filtro de média.   

\subsection{\textbf{Detecção e Localização da Placa}}

A detecção da placa é a tarefa mais desafiadora dessa abordagem, pois a placa está contida em uma pequena região da imagem, mas pode ser diferenciada pelas suas características como Cor, forma retangular e presença de caracteres. Dado uma imagem de entrada, o alvo principal da detecção é marcar uma área com probabilidade máxima de ter caracteres e validar a placa como verdadeira \cite{agarwal2016}.

Em seu trabalho, \cite{agarwal2016} explica que a detecção e a localização da placa são feitas em dois estágios:

\begin{itemize}
	\item[a)] detecção das bordas;
	\item[b)] extração da placa.
\end{itemize}

O método de detecção de bordas é realizado através do algoritmo de limiarização de Canny, que quando aplicado à imagem pré-processada, destaca todas as bordas da imagem, onde a imagem resultante depois de aplicar o método de detecção Canny é uma imagem binária com bordas realçadas. A \autoref{fig:canny} mostra a imagem após ser aplicado o algoritmo de Canny.

\begin{figure}[htb]
	\centering
	\caption{{\footnotesize Imagem após a detecção de borda.}}   % nome da figura
	\label{fig:canny}
	\includegraphics[width=.6\textwidth]{7.pdf}
	
	%\legend{Fonte: \citeonline{Vestibulando_2016}.}   % outra opção de legenda da foto
	{\footnotesize Fonte:~\citeonline{agarwal2016}.}
\end{figure}

Para identificar a área da placa, \cite{islam2015} também utilizaram o algoritmo de Canny e para classificar todo o ruído do fundo da placa e conservar a área da placa na imagem utilizou-se o filtro de média.

Na parte de extração da placa, \citeonline{agarwal2016} abordam o seguinte método de extrair todos os componentes conectados na imagem e preencher todos os buracos na imagem. Todos os buracos são preenchidos com cor branca. A \autoref{fig:placa_ext} mostra como foi feita a extração da área da placa após a localização das bordas ou contornos.

\begin{figure}[htb]
	\centering
	\caption[\footnotesize Extração da placa após a detecção de contorno.]{\footnotesize Extração da Placa após a detecção de contorno. a) Área da placa preenchida na cor branca; e b) Área da placa extraída após o processo de preenchimento.}   % nome da figura
	\label{fig:placa_ext}
	\begin{subfigure}{.4\textwidth}
		\centering
		\includegraphics[width=.9\linewidth]{8a.pdf}
		\caption{ }
	\end{subfigure}
	\begin{subfigure}{.4\textwidth}
		\centering
		\includegraphics[width=.9\linewidth]{8b.pdf}
		\caption{ }
	\end{subfigure}
	\\
	%\legend{Fonte: \citeonline{Vestibulando_2016}.}   % outra opção de legenda da foto
	{\footnotesize Fonte:~\citeonline{agarwal2016}.}
\end{figure}

Na extração da área da placa, \cite{islam2015} aplicaram operações morfológicas para remover objetos irrelevantes na imagem e por último, dilatação e erosão foram realizadas afim de extrair as áreas desejadas da placa a partir da imagem processada.

Outro método muito utilizado nos artigos revisados, foi o método do algoritmo de Sobel.  onde \citeonline{babu2016} utilizaram esse método pelo fato de que o algoritmo identifica os contornos quando há uma nítida variação na intensidade do gradiente na imagem. 

Depois de ter feito o processo de conversão da imagem para tons de cinza e converter novamente a imagem para preto e branco para que imagem ficasse somente com duas cores, foi feito a remoção dos ruídos e \emph{pixels} de valores baixos, a fim de obter uma imagem mais refinada para se ter uma visão mais limpa dos caracteres na imagem. Assim como \citeonline{khan2016,khan2016comparison} aplicaram o algoritmo de Sobel para verificar a localização exata da placa no carro e preenchendo a forma retangular conforme a \autoref{fig:sobel}a e removerá outros componentes conectados abaixo de 1000 \emph{pixels} conforme a \autoref{fig:sobel}b.

\begin{figure}[htb]
	\centering
	\caption[\footnotesize Aplicação do Algoritmo de Sobel.]{\footnotesize Aplicação do algoritmo de Sobel. a) Imagem limiarizada por Sobel; e b) Remoção de \emph{pixels}.}   % nome da figura
	\begin{subfigure}{.4\textwidth}
		\centering
		\label{fig:sobel_a}
		\includegraphics[width=.9\linewidth]{9a.pdf}
		\caption{ }
	\end{subfigure}
	\begin{subfigure}{.4\textwidth}
		\centering
		\label{fig:sobel_b}
		\includegraphics[width=.9\linewidth]{9b.pdf}
		\caption{ }
	\end{subfigure}
	\\
	%\legend{Fonte: \citeonline{Vestibulando_2016}.}   % outra opção de legenda da foto
	\label{fig:sobel}
	{\footnotesize Fonte:~\citeonline{khan2016}.}
\end{figure}

No processo de localização da placa, o trabalho de \cite{saleem2016,sen2014,ha2016} após ter convertido a imagem em escala de cinza, a imagem resultante é dilatada com elemento estruturante ou máscara. Este processo melhora as regiões brilhantes rodeadas por regiões escuras ou ao contrário, para regiões escuras rodeadas por regiões brilhantes. Assim, na presença de contornos, o efeito de dilatação da imagem é maximizado, facilitando emprego do algoritmo de Sobel.

\subsection{\textbf{Segmentação de Caracteres}} 

Após a fase de localização e extração da área de interesse (placa), inicia-se a fase de segmentação dos caracteres. Conforme \cite{george2016}, a segmentação dos caracteres é um passo importante, pois a precisão do reconhecimento do objeto depende da precisão da segmentação. O método utilizado por \citeonline{george2016} foi o método da projeção, a fim de extrair cada caractere. A localização inicial e final, superior e inferior da placa deve ser determinada. A adição das colunas individuais e das linhas individuais da placa extraída passa a projeção horizontal e vertical. O método da projeção é mostrado na \autoref{fig:projecao_method} e os caracteres segmentados são mostrados na \autoref{fig:char_seg}.

\begin{figure}[htb]
	\centering
	\caption{{\footnotesize Projeção horizontal e vertical da imagem.}}   % nome da figura
	\label{fig:projecao_method}
	\includegraphics[width=.6\textwidth]{10.pdf}
	
	%\legend{Fonte: \citeonline{Vestibulando_2016}.}   % outra opção de legenda da foto
	{\footnotesize Fonte:~\citeonline{george2016}.}
\end{figure}

\begin{figure}[htb]
	\centering
	\caption{{\footnotesize Caracteres segmentados.}}   % nome da figura
	\label{fig:char_seg}
	\includegraphics[width=.6\textwidth]{11.pdf}
	
	%\legend{Fonte: \citeonline{Vestibulando_2016}.}   % outra opção de legenda da foto
	{\footnotesize Fonte:~\citeonline{george2016}.}
\end{figure}

O trabalho de \citeonline{ikeizumie2014} também aborda o método da projeção, onde na verdade os autores descrevem duas técnicas de segmentação: o método da projeção e a detecção de contornos. O método da projeção é feito da seguinte forma: uma cadeia de caracteres é enquadrada através da projeção horizontal e vertical para haver uma separação de acordo com as suas regiões minimas da projeção vertical da imagem resultante.

Após feito o processo citado há um enquadramento dos caracteres pelas projeções. Os caracteres são separados de acordo com as suas regiões minimas da projeção vertical. \cite{ikeizumie2014} ainda descrevem que uma das grandes dificuldades de segmentar é quando ocorre uma junção indesejada causada pela proximidade entre os caracteres, onde a solução é averiguar se a largura do objeto encontrado é maior do que o máximo permitido para um caractere. Caso haja uma largura maior que o padrão o objeto é dividido ao meio.

A fim de encontrar elementos conexos, para cada \emph{pixel} $p$ de coordenadas $(x,y)$, seus 8 vizinhos são analisados. Esta conectividade está ligada a relação entre os \emph{pixels} adjacentes. Assim, os objetos com altura e largura similares a de um caractere são delimitados assim como na \autoref{fig:char_del}. 

\begin{figure}[htb]
	\centering
	\caption{{\footnotesize Caracteres delimitados através da detecção de contornos.}}   % nome da figura
	\label{fig:char_del}
	\includegraphics[width=.6\textwidth]{12.pdf}
	
	%\legend{Fonte: \citeonline{Vestibulando_2016}.}   % outra opção de legenda da foto
	{\footnotesize Fonte:~\citeonline{ikeizumie2014}.}
\end{figure}

Se sete objetos iguais a um caractere forem encontrados, a região dos sete é selecionada para o processo posterior onde os caracteres serão recortados para o reconhecimento. 

\subsection{\textbf{Reconhecimento de Caracteres}}

Após todos os processos importantes que foram descritos ao longo deste capitulo, a fase crucial para um sistema de reconhecimento automático de placas veiculares será então abordada. A fase de reconhecimento dos caracteres depende de que as outras fases anteriores sejam realizadas da melhor forma possível. Sendo assim, será visto algumas abordagens de autores já citados ao longo do capítulo.

Dos trabalhos realizados e citados, 46\% utilizaram o método de correspondência por padrão (\textit{Template Matching}) para a fase final de cada sistema. \cite{agarwal2016} argumentam que o \textit{Template Matching} é adequado quando o desvio padrão do modelo em comparação a imagem de origem for insignificante. A imagem é passada através do processo de correspondência (\textit{Matching Process}) que é realizado \emph{pixel} a \emph{pixel}. O modelo de comparação percorre todas as posições possíveis na imagem de origem resultando um valor de índice numérico que mostra o quanto o modelo sem compara a imagem original naquela posição.

A \autoref{fig:char_comp} mostra um modelo usado para a comparação no tamanho de 42x24 \emph{pixels}.

\begin{figure}[htb]
	\centering
	\caption{{\footnotesize Modelo usado para comparação.}}   % nome da figura
	\label{fig:char_comp}
	\includegraphics[width=.6\textwidth]{13.pdf}
	
	%\legend{Fonte: \citeonline{Vestibulando_2016}.}   % outra opção de legenda da foto
	{\footnotesize Fonte:~\citeonline{agarwal2016}.}
\end{figure}

Segundo \citeonline{agarwal2016}, esta técnica de correspondência é aplicada para a classificação de objetos, para a identificação de caracteres impressos, números e pequenos outros objetos. Neste método, os modelos e a imagem de origem são correlacionados. A correlação é a medida do grau de associação entre duas variáveis. A variável é basicamente os valores de \emph{pixels} correspondentes em ambas as imagens, isto é, imagem de origem e modelo. O valor de correspondência fica entre $-1$ e $+1$ aonde, quanto maior o valor de correlação, mais forte a relação entre o modelo e a imagem de origem.

No trabalho \citeonline{george2016} utilizaram \textit{Artificial Neural Networking} (ANN) para melhorar a performance do reconhecimento de caractere, já que o método de correspondência por padrão (\textit{Template Matching}) pode reconhecer somente os caracteres que mostram a similaridade com o modelo padrão criado para cada caractere no banco de dados.

Para \citeonline{trentini2010}, o algoritmo \textit{Random Trees}, também chamado pela sua coletividade de \textit{Random Forests}, o qual é um classificador baseado em árvores de decisão e que pode reconhecer os padrões de várias classes ao mesmo tempo, é composto por inúmeras árvores, formando assim florestas de decisão. Para seu trabalho esse foi o melhor método encontrado.

Outro algoritmo importante e muito utilizado no campo da visão computacional em reconhecimento de caracteres é o \textit{Tesseract tool}. \cite{jia2016} utilizaram o \textit{Tesseract}, mas em principio para seu trabalho o método de reconhecimento utilizado foi o de correspondência de padrões (\textit{Template Matching}). Porém, a taxa de reconhecimento foi de apenas 95\%, ainda tendo alguns problemas, como diferenças insuficientes entre algumas letras e dígitos para o computador reconhecer. Sendo assim foi adotado o algoritmo \textit{Tesseract}, que é um algoritmo de reconhecimento de caractere de código aberto da Google.

Isso ajuda no treinamento do processo de reconhecimento a partir dos caracteres extraídos para convertê-los em um texto que será facilmente reconhecido pelo computador. Como o classificador adaptativo da ferramenta \textit{Tesseract} recebeu alguns dados de treinamento, os caracteres extraídos podem ser convertidos em texto e reconhecidos corretamente.

\section{\uppercase{Considerações Finais}}

Neste capitulo foram vistos muitos detalhes que poderão auxiliar na solução de problemas que possam surgir na metodologia adotada.

A análise desses trabalhos contribuíram bastante para um melhor entendimento das técnicas e etapas de reconhecimento automático de placas veiculares.

Por ser uma das primeiras etapas do processo de melhoria de segurança em portões eletrônicos usando reconhecimento de padrões, foi primordial se aprofundar nas técnicas envolvidas para esse fim.

	% ----------------------------------------------------------
	
	% ----------------------------------------------------------
	% Capitulo com exemplos de comandos inseridos de arquivo externo
	% Este capitulo deve conter método proposto de seu trabalho 
	% ----------------------------------------------------------
	%\chapter{\textbf{\uppercase{Método Proposto no Reconhecimento de Placas Veiculares}}}\label{cap_4}

Nos capítulos anteriores, foram apresentadas diversas tecnologias que servirão como base teórica para o método proposto. Neste capitulo será descrito as abordagens de interligação dos métodos adotados pelos trabalhos revisados a fim de obter a resolução do problema de melhoria de segurança em portões eletrônicos utilizando reconhecimento de padrão.

\section{\uppercase{O Descrição Geral do Sistema}}

O sistema desenvolvido nesse trabalho constitui-se de um sistema de controle de acesso de automóveis onde haja a necessidade do controle de entrada em ambientes reservados. A \autoref{fig:sistema} mostra um diagrama de blocos do funcionamento do sistema que inicia desde a aquisição da imagem até a tomada de decisão do sistema. 

\begin{figure}[htb]
	\centering
	\caption{{\footnotesize Diagrama de blocos do sistema.}}   % nome da figura
	\label{fig:sistema}
	\includegraphics[width=.9\textwidth]{14.pdf}
	
	%\legend{Fonte: \citeonline{Vestibulando_2016}.}   % outra opção de legenda da foto
	{\footnotesize Fonte: Próprio Autor, 2017.}
\end{figure}

O proposito principal do sistema é fazer o reconhecimento automático de caracteres alfanuméricos na placa de matricula veicular sem a necessidade da intervenção humana para a liberação ao acesso no portão eletrônico. Inicialmente é adquirida uma imagem da placa de um carro pelo sensor ótico provido por uma \emph{webcam}. Que através de algoritmos próprios para tal função, passa para a etapa de pré-processamento da imagem (realce e redução de ruídos), justamente para melhorar a qualidade da imagem obtida. O próximo passo é extrair as características dessa imagem, ou seja, reduzir uma grande quantidade de dados que não sejam relevantes, restando somente a região de interesse. O passo seguinte será a fase de classificação ou reconhecimento do padrão, que nesse caso são os caracteres alfanuméricos.

Após o reconhecimento dos caracteres vem a fase da tomada de decisão onde através de um banco de dados ou uma lista previamente preenchida. Que irá conter dados alfanuméricos de placas que possuem acesso ou não ao ambiente fechado. Essa tomada de decisão aciona o algoritmo de acionamento do motor do portão eletrônico: se na tomada de decisão a resposta for positiva o portão será aberto, o veículo passará pelo portão e acionará um sensor ultrassônico de presença, após a passagem do veículo pelo sensor o portão se fechará, se a resposta for negativa o portão continuará fechado. A \autoref{fig:alpr} ilustra de modo geral o real funcionamento do sistema de reconhecimento de caracteres na placa de um veículo. 

\begin{figure}[htb]
	\centering
	\caption{{\footnotesize Sistema baseado em ALPR.}}   % nome da figura
	\label{fig:alpr}
	\includegraphics[width=.9\textwidth]{15.pdf}
	
	%\legend{Fonte: \citeonline{Vestibulando_2016}.}   % outra opção de legenda da foto
	{\footnotesize Fonte: Próprio Autor, 2017 (adaptado de ~\citeonline{vallapReddy2014}).}
\end{figure}

\section{\uppercase{Descrição do Algoritmo de Reconhecimento Automático de Placas Veiculares}}

O algoritmo de reconhecimento é basicamente dividido em duas partes: extração da área da placa e reconhecimento dos caracteres na área destacada, que neste caso é a placa do veiculo. Após a aquisição da imagem o primeiro passo na busca pela extração da placa é o pré-processamento.

\subsection{\textbf{Pré-processamento}}

Tido como fase inicial, o pré-processamento tem como objetivo converter a imagem adquirida no sistema de cores RGB e converte-la em tons de cinza como mostrado na~\autoref{fig:tonscinza}. Também é aplicado na imagem um filtro gaussiano que ajuda eliminar possíveis ruídos na imagem.

\begin{figure}[htb]
	\centering
	\caption[\footnotesize Conversão da imagem original para tons de cinza.]{\footnotesize Conversão da imagem original para tons de cinza. a) Imagem original; e b) Imagem em tons de cinza.}   % nome da figura
	\label{fig:tonscinza}
	\begin{subfigure}{.4\textwidth}
		\centering
		%\label{fig:sobel_a}
		\includegraphics[width=.9\linewidth]{16a.pdf}
		\caption{ }
	\end{subfigure}
	\begin{subfigure}{.4\textwidth}
		\centering
		%\label{fig:sobel_b}
		\includegraphics[width=.9\linewidth]{16b.pdf}
		\caption{ }
	\end{subfigure}
	\\
	%\legend{Fonte: \citeonline{Vestibulando_2016}.}   % outra opção de legenda da foto
	{\footnotesize Fonte: Próprio autor, 2017.}
\end{figure}

Após o processo de conversão da imagem original para tons de cinza a imagem é limiarizada (\textit{Thresholded}) criando uma imagem binarizada realçando os contornos na imagem, conforme a \autoref{fig:thresh} demonstra.

\begin{figure}[htb]
	\centering
	\caption{{\footnotesize Imagem Limiarizada/Binarizada.}}   % nome da figura
	\label{fig:thresh}
	\includegraphics[width=.7\textwidth]{17.pdf}
	
	%\legend{Fonte: \citeonline{Vestibulando_2016}.}   % outra opção de legenda da foto
	{\footnotesize Fonte: Próprio autor, 2017.}
\end{figure}

\subsection{\textbf{Extração da Placa}}

Após a fase inicial de pré-processamento, a imagem se submete a fase de extração da área da placa. Aplica-se na imagem o algoritmo de detecção de contornos deixando a imagem com os contornos mais realçados conforme a \autoref{fig:contorno} mostra.

\begin{figure}[htb]
	\centering
	\caption{{\footnotesize Imagem após a aplicação do algoritmo de detecção de contornos.}}   % nome da figura
	\label{fig:contorno}
	\includegraphics[width=.7\textwidth]{18.pdf}
	
	%\legend{Fonte: \citeonline{Vestibulando_2016}.}   % outra opção de legenda da foto
	{\footnotesize Fonte: Próprio autor, 2017.}
\end{figure}

Depois que os contornos são realçados o algoritmo detecta um conjunto de \emph{pixels} interconectados. O algoritmo processa essa área do conjunto de \emph{pixels} descartando os \emph{pixels} irrelevantes na imagem, conforme a \autoref{fig:pixels}.

\begin{figure}[htb]
	\centering
	\caption{{\footnotesize Imagem processada com conjunto de \textit{pixels} interconectados.}}   % nome da figura
	\label{fig:pixels}
	\includegraphics[width=.5\textwidth]{19.pdf}
	
	%\legend{Fonte: \citeonline{Vestibulando_2016}.}   % outra opção de legenda da foto
	{\footnotesize Fonte: Próprio autor, 2017.}
\end{figure}

O algoritmo continua com o processo de descarte dos \emph{pixels} irrelevantes deixando apenas a região de interesse da placa ficando somente a região dos caracteres. Então, é aplicado um retângulo vermelho delimitador cortando a imagem naquela região e como resultado, a extração da placa como pode ser visto na \autoref{fig:region}.

\begin{figure}[htb]
	\centering
	\caption[\footnotesize Área de interesse da imagem.]{\footnotesize Área de interesse da imagem. a) Construção do retangulo vermelho sobre a região da placa; e b) Placa extraída da imagem.}   % nome da figura
	\label{fig:region}
	\begin{subfigure}{.4\textwidth}
		\centering
		%\label{fig:sobel_a}
		\includegraphics[width=1\linewidth]{20a.pdf}
		\caption{ }
	\end{subfigure}
	\begin{subfigure}{.4\textwidth}
		\centering
		%\label{fig:sobel_b}
		\includegraphics[width=1\linewidth]{20b.pdf}
		\caption{ }
	\end{subfigure}
	\\
	%\legend{Fonte: \citeonline{Vestibulando_2016}.}   % outra opção de legenda da foto
	{\footnotesize Fonte: Próprio autor, 2017.}
\end{figure}

\subsection{\textbf{Classificação (Reconhecimento dos Caracteres)}}

Nessa etapa, depois que a área da placa foi extraída. O algoritmo refaz novamente os passos anteriores de pré-processamento, limiarização, detecção dos contornos dos caracteres, para separar individualmente os caracteres. O algoritmo aplica o método de \textit{bounding box} onde a caixa delimitadora separa os caracteres individualmente, segundo \citeonline{babu2016}: ``Quando a região especificada é rotulada, ela determina as coordenadas de canto da caixa delimitadora e também sua altura e largura''. A \autoref{fig:del_chars} apresenta a aplicação do método \textit{bounding box} na imagem.

\begin{figure}[htb]
	\centering
	\caption{{\footnotesize Caixa delimitadora nos caracteres.}}   % nome da figura
	\label{fig:del_chars}
	\includegraphics[width=.5\textwidth]{21.pdf}
	
	%\legend{Fonte: \citeonline{Vestibulando_2016}.}   % outra opção de legenda da foto
	{\footnotesize Fonte: Próprio autor, 2017.}
\end{figure}

Após esta etapa, é carregado o algoritmo de reconhecimento de caracteres, aonde o classificador escolhido foi o classificador kNN, que segundo \citeonline{rosebrock2016}, é o mais simples algoritmo de classificação de imagem e aprendizado de máquina.

O classificador kNN passou pela fase de treinamento com um conjunto de caracteres. Onde cada caractere alfanumérico de diferentes fontes, foram reconhecidos e testados em um sistema a parte, a qual constitui o dicionário de características. A \autoref{fig:trainknn} ilustra de um modo geral como funciona o classificador kNN através de seu treinamento até o reconhecimento dos caracteres ao qual ele foi treinado.

\begin{figure}[htb]
	\centering
	\caption{{\footnotesize Ilustração do funcionamento de treino do classificador kNN.}}   % nome da figura
	\label{fig:trainknn}
	\includegraphics[width=.7\textwidth]{22.pdf}
	
	%\legend{Fonte: \citeonline{Vestibulando_2016}.}   % outra opção de legenda da foto
	{\footnotesize Fonte: Próprio autor, 2017 (adaptado de~\citeonline{rosebrock2016}).}
\end{figure}

Depois de passar por todos os processos anteriores, os caracteres finalmente são reconhecidos pelo classificador e impressos na imagem, o resultado pode ser visto na \autoref{fig:print_chars}.

\begin{figure}[htb]
	\centering
	\caption{{\footnotesize Caracteres reconhecidos e inseridos na imagem.}}   % nome da figura
	\label{fig:print_chars}
	\includegraphics[width=.6\textwidth]{23.pdf}
	
	%\legend{Fonte: \citeonline{Vestibulando_2016}.}   % outra opção de legenda da foto
	{\footnotesize Fonte: Próprio autor, 2017.}
\end{figure}

\section{\uppercase{Algoritmo de acionamento do portão eletrônico}}

Neste algoritmo é utilizado como \textit{input} o resultado do algoritmo de reconhecimento de caracteres, que a nível de teste do protótipo será comparado a uma lista de placas de carro. Onde, caso os caracteres reconhecidos estejam na lista, o portão que, normalmente estará fechado, será acionado através do motor que o abrirá permitindo a passagem do veículo. Logo após a entrada do portão o sensor ultrassônico de presença é acionado medindo a distância entre o veículo e o sensor. Assim que o veículo se deslocar e passar pelo sensor, o sensor passa a medir uma distância muito maior do que a anterior e dá a permissão de comando para acionar o motor de forma que feche o portão novamente. Caso contrário, se a placa não esteja na lista, o portão permanece fechado.

\section{\uppercase{Protótipo de um portão eletrônico para simular o funcionamento do sistema}}

O protótipo deste projeto foi desenvolvido com base em um sistema de portão eletrônico utilizado em condomínios sendo movimentado por um motor DC para a abertura do portão e fechado por um sensor ultrassônico de presença, que verifica através de uma distância estipulada se o automóvel já passou (ou não) pelo portão. O protótipo corresponde a uma miniatura do sistema real. 

A \autoref{fig:proto1} demostra o protótipo utilizado para simular o funcionamento do motor e do portão eletrônico. Foi utilizado um drive de CD-ROM, já que o mesmo tem as características necessárias de um portão eletrônico real (a bandeja de CD será utilizada como um portão eletrônico e, a mesma, já contém o circuito e controle por um motor).

\begin{figure}[htb]
	\centering
	\caption{{\footnotesize Protótipo do sistema.}}   % nome da figura
	\label{fig:proto1}
	\includegraphics[width=.7\textwidth]{24.pdf}
	
	%\legend{Fonte: \citeonline{Vestibulando_2016}.}   % outra opção de legenda da foto
	{\footnotesize Fonte: Próprio autor, 2017.}
\end{figure}

%Para simular o funcionamento do motor e do portão eletrônico, foi utilizado um drive de CD-ROM, já que o mesmo tem as características necessárias de um portão eletrônico real (a bandeja de CD será utilizada como um portão eletrônico e, a mesma, já contem o circuito e controle por um motor).

A unidade controladora que executa além do processamento das imagens e o reconhecimento dos caracteres, executa o acionamento do motor e o funcionamento dos sensores é o Raspberry Pi.

Em suma, o custo base desse protótipo está listado na \autoref{tab:precos}.

\begin{table}[htb]
	\ABNTEXfontereduzida
	\caption[\footnotesize Preços dos equipamentos de Hardware investidos no projeto]{\footnotesize Preços dos equipamentos de Hardware investidos no projeto.}
	\label{tab:precos}
	\centering
	\begin{tabular}{c|c|c|c|c}
		\hline
		\textbf{\#} & \textbf{Equipamento}  & \textbf{Quantidade}  & \textbf{Preço Unitário} & \textbf{Preço Total}\\
		\hline \hline
		1 & Raspberry Pi 3 Model B & 1 & $ R\$ 299,00 $ & $ R\$ 299,00 $ \\
		\hline
		2 & \textit{Webcam} & 1 & $ R\$ 259,00 $ & $ R\$ 259,00 $ \\
		\hline
		3 & Cartão de memória de 8GB classe 10 & 1 & $ R\$ 49,00 $ & $ R\$ 49,00 $ \\
		\hline
		4 & Motor DC & 1 & $ R\$ 58,00 $ & $ R\$ 58,00 $ \\
		\hline
		5 & Fonte de alimentação 5VDC & 1 & $ R\$ 9,00 $ & $ R\$ 9,00 $ \\
		\hline
		6 & Sensor Ultrassônico & 1 & $ R\$ 30,00 $ & $ R\$ 30,00 $ \\
		\hline 
		7 & Driver de ponte H & 1 & $ R\$ 25,00 $ & $ R\$ 25,00 $ \\
		\hline  
		8 & Protoboard 20x20 & 1 & $ R\$ 20,00 $ & $ R\$ 20,00 $ \\
		\hline 
		9 & Resistores de $1k\Omega$ & 5 & $ R\$ 0,50 $ & $ R\$ 2,50 $ \\
		\hline
		10 & \textit{Jumpers}/Cabos & 50 & $ R\$ 0,30 $ & $ R\$ 15,00 $ \\ 
		\hline \hline
		\multicolumn{4}{r}{\textbf{Total}} & $ R\$ 766,50 $ \\
		\hline
	\end{tabular}
	\legend{Fonte: Próprio autor, 2017}
\end{table}

\section{\uppercase{Considerações Finais}}

Neste capitulo foi visto os passos utilizados para compor o protótipo e o projeto em si. Apesar de que alguns itens saíram bem mais caros do que o esperado, como por exemplo a \textit{webcam}. No começo foi comprado uma genérica VGA no valor de $R\$ 32,99$, a mesma se mostrou bem ineficiente em relação ao tempo de acionamento do sensor, criando um assincronismo muito grande entre o processamento do Raspberry e o sensor da câmera. Por mais que o código tenha sido refeito e otimizado para a câmera, a solução foi comprar uma câmera melhor, que no caso foi uma Logitech C525, e que encareceu um pouco o projeto.

Contudo, os resultados podem ser conferidos no próximo capitulo.

	% ----------------------------------------------------------
	
	% ----------------------------------------------------------
	% Capitulo com exemplos de comandos inseridos de arquivo externo
	% Este capitulo deve conter os resultados parciais e finais de seu trabalho 
	% ----------------------------------------------------------
	%\chapter{\textbf{\uppercase{Resultados}}}\label{cap_5}

Neste capítulo serão apresentados os resultados obtidos durante o transcorrer do projeto com base nas etapas do método proposto. 

Com o sistema em funcionamento, a permissão da aquisição da imagem se deu por meio de um sensor ultrassônico que mediu a proximidade das placas numa de média entre 50 a 60 cm de distância. Quando o sensor interpretava a média de distância estipulada, a câmera era acionada e realizava a aquisição da imagem. A mesma é enviada para o próximo passo, processamento digital da imagem, que por sua vez passava a imagem processada para as demais etapas do processo de reconhecimento da placa e dos caracteres.

Para realizar as simulações das placas de carros, foram utilizadas placas impressas em um papel cartão com as características da placa de um veiculo com placa brasileira (formato, fonte e tamanho), conforme a \autoref{fig:placa_simul}.

\begin{figure}[htb]
	\centering
	\caption{{\footnotesize Placa usada como exemplo para simular uma placa real.}}   % nome da figura
	\label{fig:placa_simul}
	\includegraphics[width=.9\textwidth]{25.pdf}
	
	%\legend{Fonte: \citeonline{Vestibulando_2016}.}   % outra opção de legenda da foto
	{\footnotesize Fonte: Próprio autor, 2017.}
\end{figure}

Para alcançar os resultados no sistema de reconhecimento, cada placa passou por um mínimo de três vezes na execução do sistema para que não houvesse dúvidas na precisão do reconhecimento. A \autoref{fig:placa_capture} mostra o momento em que a \textit{webcam} está capturando a imagem da placa sendo acionada pelo sinal que mede a distância determinada no algoritmo através do sensor ultrassônico.

\begin{figure}[htb]
	\centering
	\caption{{\footnotesize Momento da captura da imagem da placa (simulação).}}   % nome da figura
	\label{fig:placa_capture}
	\includegraphics[width=.5\textwidth]{26.pdf}
	
	%\legend{Fonte: \citeonline{Vestibulando_2016}.}   % outra opção de legenda da foto
	{\footnotesize Fonte: Próprio autor, 2017.}
\end{figure}

O tempo de execução do sistema foi de aproximadamente 10 segundos entre a aquisição da imagem a abertura do portão. O principal fator nesta demora é o acionamento do sensor da câmera e a captura da imagem (correspondendo a 63\% do tempo de operação). Em outro momento, pode-se ver através da \autoref{fig:placa_hardware} os detalhes da \textit{webcam}, do sensor ultrassônico, do Raspberry e o circuito que interliga o hardware do sistema.

\begin{figure}[htb]
	\centering
	\caption{{\footnotesize Detalhes do hardware que compõe o sistema.}}   % nome da figura
	\label{fig:placa_hardware}
	\includegraphics[width=.5\textwidth]{27.pdf}
	
	%\legend{Fonte: \citeonline{Vestibulando_2016}.}   % outra opção de legenda da foto
	{\footnotesize Fonte: Próprio autor, 2017.}
\end{figure}

A \autoref{tab:results} mostra em uma coluna as quantidades de caracteres alfanuméricos reconhecidos pelo sistema conforme a distância determinada que foi menor ou igual a 100 cm tendo os melhores resultados na média de 50 a 60 cm de distância da câmera. Além de mostrar a taxa de acerto em termos de quantos caracteres foram corretamente identificados da placa. 

No teste, foram utilizadas 26 placas diferentes nas fontes Arial e \textit{Mandatory}. Essa pontuação serve como base para a comprovação da eficiência do código de abertura do portão. E lembrando também que as condições de iluminação estavam favoráveis, devido a iluminação artificial adicionada a cena.

\begin{table}[htb]
	\ABNTEXfontereduzida
	\caption[\footnotesize Resultado do processo de reconhecimento de caracteres.]{\footnotesize Resultado do processo de reconhecimento de caracteres.}
	\label{tab:results}
	\begin{tabular}{c|c|c|c|c|c}
		\hline
		\textbf{\#} & \textbf{Placa} & \textbf{Distância (cm)} & \textbf{Caracteres Reconhecidos} & \textbf{Qtde. de Acertos} & \textbf{\%} \\
		\hline \hline
		1 & RCI-1973 & 56,75 & RCI1973 & 7 & 100\% \\
		\hline
		2 & CSC-2013 & 57,61 & CSC2013 & 7 & 100\% \\
		\hline
		3 & HQW-5678 & 57,66 & 15678 & 4 & 57\% \\
		\hline
		4 & BCD-3456 & 57,38 & BCD3456 & 7 & 100\% \\
		\hline
		5 & ABC-1234 & 46,74 & 4BC1234 & 6 & 86\% \\
		\hline
		6 & CEL-8142 & 59,01 & CEL8142 & 7 & 100\% \\
		\hline
		7 & JBS-2017 & 60,45 & JBS2017 & 7 & 100\% \\
		\hline
		8 & GAY-2424 & 58,54 & GAY2424 & 7 & 100\% \\
		\hline
		9 & SAP-5470 & 59,94 & SAP5470 & 7 & 100\% \\
		\hline
		10 & RIO-2016 & 60,36 & R2016 & 5 & 71\% \\
		\hline
		11 & ARG-0316 & 59,98 & ARG0316 & 7 & 100\% \\
		\hline
		12 & VEN-8374 & 59,13 & VEN8374 & 7 & 100\% \\
		\hline
		13 & EUA-6574 & 69,63 & EU6574 & 6 & 86\% \\
		\hline
		14 & BRA-1980 & 58,76 & BRA1980 & 7 & 100\% \\
		\hline
		15 & PTZ-1437 & 59,64 & PTZ1437 & 7 & 100\% \\
		\hline
		16 & MAN-9128 & 59,52 & JAN9128 & 6 & 86\% \\
		\hline
		17 & DIR-6456 & 59,22 & DR6456 & 6 & 86\% \\
		\hline
		18 & OAC-7473 & 58,79 & OAC7473 & 7 & 100\% \\
		\hline
		19 & JWY-3080 & 58,46 & JY3080 & 6 & 86\% \\
		\hline
		20 & NOW-2371 & 58,54 & 2371 & 4 & 57\% \\
		\hline
		21 & CMN-7472 & 58,69 & CMN7472 & 7 & 100\% \\
		\hline
		22 & NOP-3034 & 58,37 & NOP3034 & 7 & 100\% \\
		\hline
		23 & NOR-0316 & 71,85 & NOR0316 & 7 & 100\% \\
		\hline
		24 & BEE-4R22 & 65,76 & BEE4R22 & 7 & 100\% \\
		\hline
		25 & TWX-2552 & 58,03 & TX2552 & 6 & 86\% \\
		\hline
		26 & PHL-7452 & 57,94 & PHL7452 & 7 & 100\% \\
		\hline
		\multicolumn{5}{r|}{Total de confiabilidade (caracteres)} & 92,34\% \\
		\multicolumn{5}{r|}{Total de confiabilidade (placas)} & 65,38\% \\  
		\hline \hline
	\end{tabular}
	\legend{\footnotesize Fonte: Próprio autor, 2017.}
\end{table} 

Para que se possa demonstrar o resultado do reconhecimento dos caracteres da placa. A \autoref{fig:resullt_screen} mostra como aparece o resultado final para quem supervisiona o sistema em funcionamento.

\begin{figure}[htb]
	\centering
	\caption{{\footnotesize Resultado do reconhecimento no sistema.}}   % nome da figura
	\label{fig:resullt_screen}
	\includegraphics[width=.7\textwidth]{28.pdf}
	
	%\legend{Fonte: \citeonline{Vestibulando_2016}.}   % outra opção de legenda da foto
	{\footnotesize Fonte: Próprio autor, 2017.}
\end{figure}

Com base nos custos do projeto foi possível realizar uma comparação de valores com um outro sistema prontos no mercado e nacional da Intelbrás OCR LPR-ITS, visto na \autoref{fig:lpr}. A \autoref{tab:compar} demonstra o comparativo de preços do mesmo sistema em sites/estabelecimentos diferentes no Brasil.

\begin{figure}[htb]
	\centering
	\caption{{\footnotesize Sistema de reconhecimento de placas veiculares da Intelbrás.}}   % nome da figura
	\label{fig:lpr}
	\includegraphics[width=.6\textwidth]{29.pdf}
	
	%\legend{Fonte: \citeonline{Vestibulando_2016}.}   % outra opção de legenda da foto
	{\footnotesize Fonte: Google Imagens, 2017.}
\end{figure}

\begin{table}[htb]
	\ABNTEXfontereduzida
	\caption[\footnotesize Comparação de valores.]{\footnotesize Comparação de valores de produtos disponíveis no mercado atual.}
	\label{tab:compar}
	\centering
	\begin{tabular}{c|c|c|c}
		\hline
		\textbf{Site/Estabelecimento} & \textbf{Produto} & \textbf{Fabricante} & \textbf{Preço encontrado} \\
		\hline \hline
		Casas Bahia & \multirow{3}{*}{\textbf{OCR LPR-ITS}} & \multirow{3}{*}{\textbf{Intelbrás}} & R\$ 26.145,00 \\
 		\cline{1-1} 
 		\cline{4-4}
 		MercadoLivre & & & R\$ 28.884,00 \\
 		\cline{1-1} 
 		\cline{4-4}
 		FX Biometria & & & R\$ 24.900,00 \\
		\hline \hline
	\end{tabular}
	\legend{\footnotesize Fonte: Próprio autor, 2017.}
\end{table} 

De acordo com a \autoref{tab:compar}, tomou-se como base o valor mais baixo que foi de R\$ 24.900,00. A economia do sistema desenvolvido ficou em torno de 97\% mais baixo que em relação ao preço de mercado na FX Biometria. Com essa comparação pode se ter uma ideia de que o projeto torna-se economicamente viável em uma possível implantação. 
	% ----------------------------------------------------------
	
	% Se precisar, insira mais capitulos...
	
	% Finaliza a parte no bookmark do PDF
	% para que se inicie o bookmark na raiz
	% e adiciona espaço de parte no Sumário
	% ----------------------------------------------------------
	\phantompart
	
	% ----------------------------------------------------------
	% Conclusão (outro exemplo de capítulo sem numeração e presente no sumário)
	% ----------------------------------------------------------
	\chapter*[Conclusão]{Conclusão}
	\addcontentsline{toc}{chapter}{Conclusão}
	% ----------------------------------------------------------
	% ----------------------------------------------------------
% Conclusão (outro exemplo de capítulo sem numeração e presente no sumário)
% ----------------------------------------------------------
% \chapter*[Conclusão]{Conclusão}
% \addcontentsline{toc}{chapter}{Conclusão}
% ----------------------------------------------------------

O proposito deste trabalho foi o desenvolver um sistema de reconhecimento automático da placa de matricula de carro utilizando reconhecimento de padrões para melhoria de segurança em portões eletrônicos.

Com esta finalidade o projeto foi concebido para fazer com que através de uma Webcam comum, um sensor ultrassônico, um pequeno motor DC (em um drive de CDROM) e um microcomputador de pequeno porte que pudessem simular e executar um sistema de reconhecimento de padrão e controlar todo o hardware empregado no sistema, podendo ser utilizado para controlar o acesso de entrada em ambientes fechados.

Para a criação do sistema foram implementados e adaptados algoritmos de reconhecimento de padrão e o de acionamento de abertura e fechamento do portão, foi feito um protótipo simples com os hardwares já descritos.

O sistema de reconhecimento em alguns momentos apresentou falhas ao tentar reconhecer caracteres como por exemplo a letra 'W'. A distância também foi um fator que contribuiu muito para a ineficácia do reconhecimento e o posicionamento da câmera em certos ângulos (maiores que 30\textdegree).
 
A câmera assim como o sensor têm que estar em uma posição estável sem movimentos ou vibrações. Notou-se que quando a haste da câmera junto com o sensor era movida, havia falha ou no sensor ou na aquisição da imagem. Outro fator importante que se pode perceber no comportamento do sistema, é que geralmente as falhas se davam em relação as letras e minimamente nos números. Outro problema identificado foi que a fonte utilizada para o treinamento do algoritmo de reconhecimento há de ser a mesma fonte que se deseja identificar, inicialmente quando foram praticados os testes, as fontes utilizadas nas placas eram da fonte \emph{Mandatory} (que é o padrão de fonte em placas veiculares no Brasil) e com essa fonte o sistema apresentou muitas falhas. Ao troca-la pela fonte Arial o sistema passou a reconhecer melhor os caracteres na placa. 	Pode-se então constatar que havia sido feito um treinamento no classificador kNN com a fonte Arial.

Na questão da identificação da área da placa o sistema conseguiu identificar em 100\% a região da área de placa. Porém como as placas utilizadas foram impressas em folhas de papel cartão para simular uma placa real, essas placas seguiram um mesmo padrão, tornando assim mais fáceis de identifica-las, o que talvez não aconteceria em uma situação real por causa do posicionamento das placas nos carros.

Em outra situação importante, destaca-se a velocidade de execução do sistema cujo o tempo de execução foi aceitável, sendo executado numa média de 10 segundos em todo o processo.

Para trabalhos futuros é preciso fazer algumas melhorias como implementar um banco de dados no sistema, fazer uma melhoria no treinamento do classificador kNN para que se possa resolver essas falhas no reconhecimento de caracteres, testar outras bibliotecas de reconhecimento de caracteres online, como por exemplo o uso do \textit{Google Tesseract OCR}, ou outros classificadores que sejam convenientes ao projeto.

Apesar deste problemas levantados na execução do sistema, os resultados comprovam uma taxa de acerto de mais de 92\% que constitui uma boa aplicação em termos de reconhecimento. Levando assim, o projeto a um sucesso razoável e capaz de cumprir os seus objetivos. 

Por fim, testar o sistema em uma situação real após a consolidação dos testes no protótipo e após todas as melhorias e ajustes serem realizados.




	
	% ----------------------------------------------------------
	% ELEMENTOS PÓS-TEXTUAIS
	% ----------------------------------------------------------
	\postextual
	% ----------------------------------------------------------
	
	% ----------------------------------------------------------
	% Referências bibliográficas
	% ----------------------------------------------------------
	\bibliography{elementos-postextuais/referencias}
	
	% ----------------------------------------------------------
	% Glossário
	% ----------------------------------------------------------
	%
	% Consulte o manual da classe abntex2 para orientações sobre o glossário.
	%
	%\glossary
	
	% ----------------------------------------------------------
	% Apêndices
	% ----------------------------------------------------------
	%% ----------------------------------------------------------
% Apêndices
% ----------------------------------------------------------

% Inicia os apêndices
\begin{apendicesenv}

\renewcommand{\lstlistingname}{Código}
\chapter{Código Fonte do Programa de Classificação}
\label{CodigoA}

\textbf{\color{orange}Definição:} Programa feito no MATLAB para receber a base de imagens, treina-las e compara-las pelo classificador.

\definecolor{mygreen}{rgb}{0,0.6,0}
\definecolor{mygray}{rgb}{0.5,0.5,0.5}
\definecolor{mymauve}{rgb}{0.58,0,0.82}

\lstset{ %
	backgroundcolor=\color{white},   % choose the background color; you must add \usepackage{color} or \usepackage{xcolor}
	basicstyle=\sffamily\scriptsize,        % the size of the fonts that are used for the code
	breakatwhitespace=true,         % sets if automatic breaks should only happen at whitespace
	breaklines=true,                 % sets automatic line breaking
	captionpos=t,                    % sets the caption-position to bottom
	commentstyle=\color{green},    	 % comment style
	deletekeywords={...},            % if you want to delete keywords from the given language
	escapeinside={\%*}{*)},          % if you want to add LaTeX within your code
	extendedchars=true,              % lets you use non-ASCII characters; for 8-bits encodings only, does not work with UTF-8
	frame=none,	                   % adds a frame around the code
	inputencoding=utf8,
	keepspaces=true,                 % keeps spaces in text, useful for keeping indentation of code (possibly needs columns=flexible)
	keywordstyle=\color{blue},       % keyword style
	language=Matlab,	                 % the language of the code
	morecomment=[l][\color{magenta}]{\#}
	otherkeywords={*,...},           % if you want to add more keywords to the set
	numbers=left,                    % where to put the line-numbers; possible values are (none, left, right)
	numbersep=5pt,                   % how far the line-numbers are from the code
	numberstyle=\tiny\color{gray}, % the style that is used for the line-numbers
	rulecolor=\color{black},         % if not set, the frame-color may be changed on line-breaks within not-black text (e.g. comments (green here))
	showspaces=false,                % show spaces everywhere adding particular underscores; it overrides 'showstringspaces'
	showstringspaces=false,          % underline spaces within strings only
	showtabs=false,                  % show tabs within strings adding particular underscores
	stepnumber=1,                    % the step between two line-numbers. If it's 1, each line will be numbered
	stringstyle=\color{red},     % string literal style
	tabsize=2,	                   % sets default tabsize to 2 spaces
	%title=\lstname,                   % show the filename of files included with \lstinputlisting; also try caption instead of title
	literate=%
	{é}{{\'{e}}}1%
	{è}{{\`{e}}}1%
	{ê}{{\^{e}}}1%
	{ë}{{\¨{e}}}1%
	{É}{{\'{E}}}1%
	{Ê}{{\^{E}}}1%
	{û}{{\^{u}}}1%
	{ù}{{\`{u}}}1%
	{ú}{{\'{u}}}1%
	{â}{{\^{a}}}1%
	{à}{{\`{a}}}1%
	{á}{{\'{a}}}1%
	{ã}{{\~{a}}}1%
	{Á}{{\'{A}}}1%
	{Â}{{\^{A}}}1%
	{Ã}{{\~{A}}}1%
	{ç}{{\c{c}}}1%
	{Ç}{{\c{C}}}1%
	{õ}{{\~{o}}}1%
	{ó}{{\'{o}}}1%
	{ô}{{\^{o}}}1%
	{Õ}{{\~{O}}}1%
	{Ó}{{\'{O}}}1%
	{Ô}{{\^{O}}}1%
	{î}{{\^{i}}}1%
	{Î}{{\^{I}}}1%
	{í}{{\'{i}}}1%
	{Í}{{\~{Í}}}1%,
}

\section*{Código Principal}

\lstinputlisting{algoritmofinal.m}

\section*{Código do Extrator}

\lstinputlisting{feat.m}

\end{apendicesenv}

	
	
	% ----------------------------------------------------------
	% Anexos
	% ----------------------------------------------------------
	%% ----------------------------------------------------------
% Anexos
% ----------------------------------------------------------

% Inicia os anexos
\begin{anexosenv}

% Imprime uma página indicando o início dos anexos
\partanexos

\chapter{Morbi ultrices rutrum lorem.}
\lipsum[30]

\chapter{Cras non urna sed feugiat cum sociis natoque penatibus et magnis dis
parturient montes nascetur ridiculus mus}

\lipsum[31]

\chapter{Fusce facilisis lacinia dui}

\lipsum[32]

\end{anexosenv}


	
	%---------------------------------------------------------------------
	% INDICE REMISSIVO
	%---------------------------------------------------------------------
	\phantompart
	\printindex
	%---------------------------------------------------------------------
	
\end{document}