% ----------------------------------------------------------
% PACOTES BÁSICOS
% ----------------------------------------------------------
\usepackage{lmodern}            % Usa a fonte Latin Modern          
\usepackage[T1]{fontenc}        % Selecao de codigos de fonte.
\usepackage[utf8]{inputenc}     % Codificacao do documento (conversão automática dos acentos)
\usepackage{lastpage}           % Usado pela Ficha catalográfica
\usepackage{indentfirst}        % Indenta o primeiro parágrafo de cada seção.
\usepackage{color}      		% Controle das cores
\usepackage{graphicx}           % Inclusão de gráficos
\usepackage{microtype}          % para melhorias de justificação
\usepackage{algorithm,algorithmic}          % Inserir código de linguagem de programação
\usepackage{float,array,multicol,multirow,booktabs,hhline,colortbl}		% pacotes para ajuda em formatação de tabelas
\usepackage{subcaption}			% para colocar multiplas imagens uma ao lado da outra
\usepackage[table]{xcolor}		% para dar stripes nas tabelas
\usepackage{amsmath}            % Pacote usado para aumentar as opções de simbolos em equações
\usepackage{enumitem}			% Usado para enumerar e melhorar as listas
\usepackage{listingsutf8}		% Serve para colocar códigos fonte com caracteres em UTF8, necessário em alguns tipos de scripts
\usepackage{textcomp}			% Serve para colocar simbolos especiais
\usepackage{gensymb}			% Serve para colocar simbolos especiais no modo matemático
%\usepackage[none]{hyphenat} 	% Serve para eliminar a hifenização no documento, que é proibido segundo a noram NBR 14724/11 
% descomente a linha acima APENAS se não houver problemas com texto passando a margem

% Pacotes de citações
\usepackage[brazilian,hyperpageref]{backref}     % Paginas com as citações na bibl
\usepackage[alf]{abntex2cite}   % Citações padrão ABNT

% CONFIGURAÇÕES DE PACOTES
% Configurações do pacote backref
% Usado sem a opção hyperpageref de backref
\renewcommand{\backrefpagesname}{Citado na(s) página(s):~}
% Texto padrão antes do número das páginas
\renewcommand{\backref}{}
% Define os textos da citação
\renewcommand*{\backrefalt}[4]{
    \ifcase #1 %
        Nenhuma citação no texto.%
    \or
        Citado na página #2.%
    \else
        Citado #1 vezes nas páginas #2.%
    \fi}%

% Pacotes adicionais, usados apenas no âmbito do Modelo Canônico do abnteX2
\usepackage{lipsum}             % para geração de dummy text

% TODO inserir seus pacotes aqui

