% ----------------------------------------------------------
% Apêndices
% ----------------------------------------------------------

% Inicia os apêndices
\begin{apendicesenv}

\renewcommand{\lstlistingname}{Código}
\chapter{Código Fonte do Programa de Classificação}
\label{CodigoA}

\textbf{\color{orange}Definição:} Programa feito no MATLAB para receber a base de imagens, treina-las e compara-las pelo classificador.

\definecolor{mygreen}{rgb}{0,0.6,0}
\definecolor{mygray}{rgb}{0.5,0.5,0.5}
\definecolor{mymauve}{rgb}{0.58,0,0.82}

\lstset{ %
	backgroundcolor=\color{white},   % choose the background color; you must add \usepackage{color} or \usepackage{xcolor}
	basicstyle=\sffamily\scriptsize,        % the size of the fonts that are used for the code
	breakatwhitespace=true,         % sets if automatic breaks should only happen at whitespace
	breaklines=true,                 % sets automatic line breaking
	captionpos=t,                    % sets the caption-position to bottom
	commentstyle=\color{green},    	 % comment style
	deletekeywords={...},            % if you want to delete keywords from the given language
	escapeinside={\%*}{*)},          % if you want to add LaTeX within your code
	extendedchars=true,              % lets you use non-ASCII characters; for 8-bits encodings only, does not work with UTF-8
	frame=none,	                   % adds a frame around the code
	inputencoding=utf8,
	keepspaces=true,                 % keeps spaces in text, useful for keeping indentation of code (possibly needs columns=flexible)
	keywordstyle=\color{blue},       % keyword style
	language=Matlab,	                 % the language of the code
	morecomment=[l][\color{magenta}]{\#}
	otherkeywords={*,...},           % if you want to add more keywords to the set
	numbers=left,                    % where to put the line-numbers; possible values are (none, left, right)
	numbersep=5pt,                   % how far the line-numbers are from the code
	numberstyle=\tiny\color{gray}, % the style that is used for the line-numbers
	rulecolor=\color{black},         % if not set, the frame-color may be changed on line-breaks within not-black text (e.g. comments (green here))
	showspaces=false,                % show spaces everywhere adding particular underscores; it overrides 'showstringspaces'
	showstringspaces=false,          % underline spaces within strings only
	showtabs=false,                  % show tabs within strings adding particular underscores
	stepnumber=1,                    % the step between two line-numbers. If it's 1, each line will be numbered
	stringstyle=\color{red},     % string literal style
	tabsize=2,	                   % sets default tabsize to 2 spaces
	%title=\lstname,                   % show the filename of files included with \lstinputlisting; also try caption instead of title
	literate=%
	{é}{{\'{e}}}1%
	{è}{{\`{e}}}1%
	{ê}{{\^{e}}}1%
	{ë}{{\¨{e}}}1%
	{É}{{\'{E}}}1%
	{Ê}{{\^{E}}}1%
	{û}{{\^{u}}}1%
	{ù}{{\`{u}}}1%
	{ú}{{\'{u}}}1%
	{â}{{\^{a}}}1%
	{à}{{\`{a}}}1%
	{á}{{\'{a}}}1%
	{ã}{{\~{a}}}1%
	{Á}{{\'{A}}}1%
	{Â}{{\^{A}}}1%
	{Ã}{{\~{A}}}1%
	{ç}{{\c{c}}}1%
	{Ç}{{\c{C}}}1%
	{õ}{{\~{o}}}1%
	{ó}{{\'{o}}}1%
	{ô}{{\^{o}}}1%
	{Õ}{{\~{O}}}1%
	{Ó}{{\'{O}}}1%
	{Ô}{{\^{O}}}1%
	{î}{{\^{i}}}1%
	{Î}{{\^{I}}}1%
	{í}{{\'{i}}}1%
	{Í}{{\~{Í}}}1%,
}

\section*{Código Principal}

\lstinputlisting{algoritmofinal.m}

\section*{Código do Extrator}

\lstinputlisting{feat.m}

\end{apendicesenv}

